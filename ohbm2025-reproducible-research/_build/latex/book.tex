%% Generated by Sphinx.
\def\sphinxdocclass{jupyterBook}
\documentclass[letterpaper,10pt,english]{jupyterBook}
\ifdefined\pdfpxdimen
   \let\sphinxpxdimen\pdfpxdimen\else\newdimen\sphinxpxdimen
\fi \sphinxpxdimen=.75bp\relax
\ifdefined\pdfimageresolution
    \pdfimageresolution= \numexpr \dimexpr1in\relax/\sphinxpxdimen\relax
\fi
%% let collapsible pdf bookmarks panel have high depth per default
\PassOptionsToPackage{bookmarksdepth=5}{hyperref}
%% turn off hyperref patch of \index as sphinx.xdy xindy module takes care of
%% suitable \hyperpage mark-up, working around hyperref-xindy incompatibility
\PassOptionsToPackage{hyperindex=false}{hyperref}
%% memoir class requires extra handling
\makeatletter\@ifclassloaded{memoir}
{\ifdefined\memhyperindexfalse\memhyperindexfalse\fi}{}\makeatother

\PassOptionsToPackage{booktabs}{sphinx}
\PassOptionsToPackage{colorrows}{sphinx}

\PassOptionsToPackage{warn}{textcomp}

\catcode`^^^^00a0\active\protected\def^^^^00a0{\leavevmode\nobreak\ }
\usepackage{cmap}
\usepackage{fontspec}
\defaultfontfeatures[\rmfamily,\sffamily,\ttfamily]{}
\usepackage{amsmath,amssymb,amstext}
\usepackage{polyglossia}
\setmainlanguage{english}



\setmainfont{FreeSerif}[
  Extension      = .otf,
  UprightFont    = *,
  ItalicFont     = *Italic,
  BoldFont       = *Bold,
  BoldItalicFont = *BoldItalic
]
\setsansfont{FreeSans}[
  Extension      = .otf,
  UprightFont    = *,
  ItalicFont     = *Oblique,
  BoldFont       = *Bold,
  BoldItalicFont = *BoldOblique,
]
\setmonofont{FreeMono}[
  Extension      = .otf,
  UprightFont    = *,
  ItalicFont     = *Oblique,
  BoldFont       = *Bold,
  BoldItalicFont = *BoldOblique,
]



\usepackage[Bjarne]{fncychap}
\usepackage[,numfigreset=1,mathnumfig]{sphinx}

\fvset{fontsize=\small}
\usepackage{geometry}


% Include hyperref last.
\usepackage{hyperref}
% Fix anchor placement for figures with captions.
\usepackage{hypcap}% it must be loaded after hyperref.
% Set up styles of URL: it should be placed after hyperref.
\urlstyle{same}


\usepackage{sphinxmessages}



        % Start of preamble defined in sphinx-jupyterbook-latex %
         \usepackage[Latin,Greek]{ucharclasses}
        \usepackage{unicode-math}
        % fixing title of the toc
        \addto\captionsenglish{\renewcommand{\contentsname}{Contents}}
        \hypersetup{
            pdfencoding=auto,
            psdextra
        }
        % End of preamble defined in sphinx-jupyterbook-latex %
        

\title{OHBM 2025 Educational; From Code to Visualization: Reproducible Pipelines for Neuroimaging Research}
\date{Jun 09, 2025}
\release{}
\author{Sina Mansour L.\@{}}
\newcommand{\sphinxlogo}{\vbox{}}
\renewcommand{\releasename}{}
\makeindex
\begin{document}

\pagestyle{empty}
\sphinxmaketitle
\pagestyle{plain}
\sphinxtableofcontents
\pagestyle{normal}
\phantomsection\label{\detokenize{intro::doc}}


\sphinxAtStartPar
From Code to Visualization

\sphinxAtStartPar
\sphinxstylestrong{Author/Presenter}:\\
\sphinxhref{https://sina-mansour.github.io/}{\sphinxstyleemphasis{Sina Mansour L.}}, Ph.D.\\
National University of Singapore \& The University of Melbourne

\begin{sphinxadmonition}{note}{Note:}
\sphinxAtStartPar
This Jupyter Book contains supporting material for the session titled \sphinxstyleemphasis{“From Code to Visualization: Reproducible Pipelines for Neuroimaging Research”} that was part of an educational course on \sphinxhref{https://ww6.aievolution.com/hbm2501/Events/viewEv?ev=2237}{“Maximizing scientific efficiency through sustainability, reproducibility, and FAIRness”}, presented at the \sphinxhref{https://www.humanbrainmapping.org/i4a/pages/index.cfm?pageid=4229}{2025 OHBM Annual Meeting in Brisbane}.
\end{sphinxadmonition}


\bigskip\hrule\bigskip


\begin{DUlineblock}{0em}
\item[] \sphinxstylestrong{\Large 💬 Synopsis}
\end{DUlineblock}

\sphinxAtStartPar
How can neuroimaging researchers ensure their findings are not only robust but also \sphinxstylestrong{transparent} and \sphinxstylestrong{reproducible}?

\sphinxAtStartPar
This book guides you through building research workflows with reproducible \sphinxstylestrong{code} and \sphinxstylestrong{visualizations}, designed to accompany your manuscript and support open science.

\begin{DUlineblock}{0em}
\item[] \sphinxstylestrong{\large 🧠 Why Reproducibility?}
\end{DUlineblock}

\sphinxAtStartPar
Reproducibility helps:
\begin{itemize}
\item {} 
\sphinxAtStartPar
✅ Maximize trust in your work

\item {} 
\sphinxAtStartPar
🤝 Enable collaboration across teams

\item {} 
\sphinxAtStartPar
🔍 Improve transparency and accountability

\end{itemize}


\bigskip\hrule\bigskip


\sphinxAtStartPar
We provide examples and practical tips to boost reproducibility:

\begin{DUlineblock}{0em}
\item[] \sphinxstylestrong{\large 💻 In your Code:}
\end{DUlineblock}
\begin{itemize}
\item {} 
\sphinxAtStartPar
📝 Write clear, shareable scripts

\item {} 
\sphinxAtStartPar
🌟 Follow best practices for sharing code

\item {} 
\sphinxAtStartPar
🌐 Use open notebooks for full pipeline visibility

\end{itemize}

\begin{DUlineblock}{0em}
\item[] \sphinxstylestrong{\large 🎨 In your Visualizations:}
\end{DUlineblock}
\begin{itemize}
\item {} 
\sphinxAtStartPar
🖼️ Understand why reproducible figures matter

\item {} 
\sphinxAtStartPar
🔁 Learn how to make visualizations easy to recreate

\item {} 
\sphinxAtStartPar
🛠️ Build scriptable visualizations for:
\begin{itemize}
\item {} 
\sphinxAtStartPar
🧠 Cortical surface projections

\item {} 
\sphinxAtStartPar
🧊 Volumetric plots

\item {} 
\sphinxAtStartPar
〰️🧵 Tractography results

\item {} 
\sphinxAtStartPar
🕸️ Brain network diagrams

\item {} 
\sphinxAtStartPar
… \& more!

\end{itemize}

\end{itemize}



\begin{DUlineblock}{0em}
\item[] \sphinxstylestrong{\large 🗂️ Table of contents}
\end{DUlineblock}

\sphinxAtStartPar
Overview of the content of this book:
\begin{itemize}
\item {} 
\sphinxAtStartPar
{\hyperref[\detokenize{chapters/01/reproducibility-in-neuroimaging::doc}]{\sphinxcrossref{📗 Why Reproducibility Matters}}}

\item {} 
\sphinxAtStartPar
{\hyperref[\detokenize{chapters/02/reproducible-analysis-pipelines::doc}]{\sphinxcrossref{📘 Building Reproducible Analysis Pipelines}}}

\item {} 
\sphinxAtStartPar
{\hyperref[\detokenize{chapters/03/reproducible-visualizations::doc}]{\sphinxcrossref{📙 Reproducible Scientific Visualizations}}}

\item {} 
\sphinxAtStartPar
{\hyperref[\detokenize{chapters/04/references::doc}]{\sphinxcrossref{📑 References}}}

\end{itemize}

\sphinxstepscope


\chapter{📗 Reproducibility in Neuroimaging}
\label{\detokenize{chapters/01/reproducibility-in-neuroimaging:reproducibility-in-neuroimaging}}\label{\detokenize{chapters/01/reproducibility-in-neuroimaging::doc}}
\sphinxAtStartPar
This chapter aims to highlight the importance of reproducible research practices and briefly explain what will be covered in the rest of the book.


\section{🤨 Why Care about Reproducibility?}
\label{\detokenize{chapters/01/reproducibility-in-neuroimaging:why-care-about-reproducibility}}\begin{itemize}
\item {} 
\sphinxAtStartPar
\sphinxstylestrong{Science builds on science}, without reproducibility, findings can’t be trusted or extended.

\item {} 
\sphinxAtStartPar
In computational neuroscience and neuroimaging, findings are often results of complex pipelines, built with multiple software tools, and parameter choices.

\item {} 
\sphinxAtStartPar
Without access to code, even small analytical tweaks can become untraceable.

\end{itemize}
\begin{quote}

\sphinxAtStartPar
“Science is a social enterprise: independent and collaborative groups work to accumulate knowledge as a public good.”

\sphinxAtStartPar
\sphinxstyleemphasis{Munafo et al. (2017)} %
\begin{footnote}[1]\sphinxAtStartFootnote
Marcus R Munafò, Brian A Nosek, Dorothy VM Bishop, Katherine S Button, Christopher D Chambers, Nathalie Percie du Sert, Uri Simonsohn, Eric\sphinxhyphen{}Jan Wagenmakers, Jennifer J Ware, and John PA Ioannidis. A manifesto for reproducible science. \sphinxstyleemphasis{Nature human behaviour}, 1(1):0021, 2017.
%
\end{footnote}
\end{quote}


\section{🤔 What’s at Stake?}
\label{\detokenize{chapters/01/reproducibility-in-neuroimaging:what-s-at-stake}}\begin{itemize}
\item {} 
\sphinxAtStartPar
\sphinxstylestrong{Credibility} of research outputs.

\end{itemize}
\begin{quote}

\sphinxAtStartPar
“The authors of research papers have no obligation to share their data and code, and I have no obligation to believe anything they write.”

\sphinxAtStartPar
\sphinxhref{https://statmodeling.stat.columbia.edu/2023/09/10/the-authors-of-research-papers-have-no-obligation-to-share-their-data-and-code-and-i-have-no-obligation-to-believe-anything-they-write/}{\sphinxstyleemphasis{Andrew Gelman}} (professor of statistics and political science at Columbia University)
\end{quote}
\begin{itemize}
\item {} 
\sphinxAtStartPar
\sphinxstylestrong{Wasted time} taken to re\sphinxhyphen{}implement procedures reported in previous works.

\end{itemize}


\section{🔀 Categorizing Reproducibility:}
\label{\detokenize{chapters/01/reproducibility-in-neuroimaging:categorizing-reproducibility}}
\sphinxAtStartPar
Botvinik\sphinxhyphen{}Nezer and Wager%
\begin{footnote}[2]\sphinxAtStartFootnote
Rotem Botvinik\sphinxhyphen{}Nezer and Tor D Wager. Reproducibility in neuroimaging analysis: challenges and solutions. \sphinxstyleemphasis{Biological Psychiatry: Cognitive Neuroscience and Neuroimaging}, 8(8):780–788, 2023.
%
\end{footnote} identify three types of reproducibility:
\begin{enumerate}
\sphinxsetlistlabels{\arabic}{enumi}{enumii}{}{.}%
\item {} 
\sphinxAtStartPar
\sphinxstylestrong{Analytical reproducibility}: Reproducing findings using the \sphinxstyleemphasis{same data} and \sphinxstyleemphasis{same methods}.

\item {} 
\sphinxAtStartPar
\sphinxstylestrong{Replicability}: Finding similar results in \sphinxstyleemphasis{independent datasets} using similar methods.

\item {} 
\sphinxAtStartPar
\sphinxstylestrong{Robustness to analytical variability}: Obtaining consistent results using \sphinxstyleemphasis{different analytical approaches}.

\end{enumerate}

\begin{sphinxadmonition}{tip}{Tip:}
\sphinxAtStartPar
The goal of this session is to introduce practices for ensuring analytical reproducibility. This can serve as a foundation for achieving replicability and methodological robustness.
\end{sphinxadmonition}

\sphinxAtStartPar
Gorgolewski and Poldrack%
\begin{footnote}[3]\sphinxAtStartFootnote
Krzysztof J Gorgolewski and Russell A Poldrack. A practical guide for improving transparency and reproducibility in neuroimaging research. \sphinxstyleemphasis{PLoS biology}, 14(7):e1002506, 2016.
%
\end{footnote} cover 3 major topics in open science (see \hyperref[\detokenize{chapters/01/reproducibility-in-neuroimaging:os-pillars}]{Fig.\@ \ref{\detokenize{chapters/01/reproducibility-in-neuroimaging:os-pillars}}}), with implications for reproducibility:
\begin{enumerate}
\sphinxsetlistlabels{\arabic}{enumi}{enumii}{}{.}%
\item {} 
\sphinxAtStartPar
\sphinxstylestrong{Data}: Access to the original data is required to examine analytical reproducibility.

\item {} 
\sphinxAtStartPar
\sphinxstylestrong{Code}: Access to the implementation scripts is also needed.

\item {} 
\sphinxAtStartPar
\sphinxstylestrong{Papers}: Access to documentations of methods, and interpretations of results.

\end{enumerate}

\begin{figure}[htbp]
\centering
\capstart

\noindent\sphinxincludegraphics[height=400\sphinxpxdimen]{{pillars}.png}
\caption{Three pillars of open science.\sphinxfootnotemark[3]}\label{\detokenize{chapters/01/reproducibility-in-neuroimaging:os-pillars}}\end{figure}

\begin{sphinxadmonition}{tip}{Tip:}
\sphinxAtStartPar
In this session, we’ll cover topics addressing \sphinxstylestrong{code} openness.
\end{sphinxadmonition}


\section{🚩 Let’s Start}
\label{\detokenize{chapters/01/reproducibility-in-neuroimaging:let-s-start}}
\sphinxAtStartPar
Now that we’ve covered why reproducibility matters and what this session will include, let’s jump in.
We’ll focus on two main topics:
\begin{itemize}
\item {} 
\sphinxAtStartPar
\sphinxstylestrong{{\hyperref[\detokenize{chapters/02/reproducible-analysis-pipelines::doc}]{\sphinxcrossref{\DUrole{doc}{📘 Building Reproducible Analysis Pipelines}}}}}

\item {} 
\sphinxAtStartPar
\sphinxstylestrong{{\hyperref[\detokenize{chapters/03/reproducible-visualizations::doc}]{\sphinxcrossref{\DUrole{doc}{📙 Reproducible Scientific Visualizations}}}}}

\end{itemize}


\bigskip\hrule\bigskip



\section{📑 References}
\label{\detokenize{chapters/01/reproducibility-in-neuroimaging:references}}

\bigskip\hrule\bigskip


\sphinxstepscope


\chapter{📘 Building Reproducible Analysis Pipelines}
\label{\detokenize{chapters/02/reproducible-analysis-pipelines:building-reproducible-analysis-pipelines}}\label{\detokenize{chapters/02/reproducible-analysis-pipelines::doc}}
\sphinxAtStartPar
This chapter aims to set the foundation for how to achieve analytical reproducibility, especially via \sphinxstylestrong{good coding practices}, \sphinxstylestrong{structured workflows}, and \sphinxstylestrong{tools} tailored for neuroimaging.

\sphinxAtStartPar
\sphinxstylestrong{Key goals of this chapter:}
\begin{itemize}
\item {} 
\sphinxAtStartPar
Show that reproducibility starts with how we write, document, and share code.

\item {} 
\sphinxAtStartPar
Emphasize modular, readable, and shareable analysis pipelines.

\item {} 
\sphinxAtStartPar
Provide examples on how to implement best practices in sharing code.

\item {} 
\sphinxAtStartPar
Introduce relevant tools that can help you in sharing your analysis code.

\end{itemize}

\sphinxstepscope


\section{📘1️⃣ Coding Best Practices}
\label{\detokenize{chapters/02/02a_coding-best-practices:coding-best-practices}}\label{\detokenize{chapters/02/02a_coding-best-practices::doc}}
\sphinxAtStartPar
Let’s address the elephant in the room: many researchers hesitate to share their code, often out of fear of being judged for how it’s written.

\sphinxAtStartPar
While some level of imposter syndrome is inevitable, there are simple, effective habits you can adopt to boost both your confidence and your code quality. These best practices don’t require much extra time, and they can save you significant effort down the line.

\sphinxAtStartPar
Code sharing is no longer a luxury, it’s increasingly expected, made easier by accessible tools and community\sphinxhyphen{}driven standards. Open science is gaining momentum, and with it, a welcome shift: code doesn’t need to be flawless to be valuable. In fact, any effort to share code is generally appreciated far more than not sharing anything at all.

\sphinxAtStartPar
In this chapter, we’ll go over a set of practical recommendations, drawn from widely accepted coding standards—that can help you prepare code you’ll feel comfortable sharing.

\sphinxAtStartPar
Not every point will apply to every project, but most will. Aim to adopt the practices that best fit your needs and workflow.


\subsection{👣 Small Steps Toward Better Coding Practices}
\label{\detokenize{chapters/02/02a_coding-best-practices:small-steps-toward-better-coding-practices}}
\sphinxAtStartPar
In the ensuing sections, we’ll dive through things you could do make your code better for sharing:


\subsubsection{📝 Commenting}
\label{\detokenize{chapters/02/02a_coding-best-practices:commenting}}
\begin{figure}[htbp]
\centering
\capstart

\noindent\sphinxincludegraphics[height=380\sphinxpxdimen]{{code_comments}.jpg}
\caption{Comments are essential for making code shareable and understandable.}\label{\detokenize{chapters/02/02a_coding-best-practices:comment-fig}}\end{figure}
\begin{itemize}
\item {} 
\sphinxAtStartPar
\sphinxstylestrong{Top\sphinxhyphen{}level script comments:} At the top of each executable script, include a short explanatory comment describing how it should be run. No matter how brief, this should include at least one example of expected usage (e.g., a sample command line call).

\item {} 
\sphinxAtStartPar
\sphinxstylestrong{Inline comments:} Throughout your code, include comments explaining what key blocks or lines are doing. This isn’t just for others—it’ll help your future self, too! 😅

\end{itemize}


\bigskip\hrule\bigskip



\subsubsection{☢️ Variables}
\label{\detokenize{chapters/02/02a_coding-best-practices:variables}}\begin{itemize}
\item {} 
\sphinxAtStartPar
\sphinxstylestrong{Avoid hard\sphinxhyphen{}coding values.} Instead of embedding parameters directly into your code (e.g., file paths, thresholds, or flags), assign them to variables.

\item {} 
\sphinxAtStartPar
\sphinxstylestrong{Centralize configurable parameters.} Define variables that others might need to modify, such as input/output paths or settings, at the top of your script. This makes your code easier to read, test, and reuse.

\end{itemize}

\begin{sphinxadmonition}{note}{Example}

\sphinxAtStartPar
Here’s an example of what \sphinxstylestrong{not} to do:

\begin{sphinxVerbatim}[commandchars=\\\{\}]
\PYG{k}{if} \PYG{n}{p\PYGZus{}value} \PYG{o}{\PYGZlt{}} \PYG{l+m+mf}{0.05}\PYG{p}{:}
    \PYG{o}{.}\PYG{o}{.}\PYG{o}{.}
\end{sphinxVerbatim}

\sphinxAtStartPar
A better approach is to define parameters as variables at the top of your script:

\begin{sphinxVerbatim}[commandchars=\\\{\}]
\PYG{c+c1}{\PYGZsh{} at the top of the script}
\PYG{n}{p\PYGZus{}value\PYGZus{}threshold} \PYG{o}{=} \PYG{l+m+mf}{0.05}

\PYG{c+c1}{\PYGZsh{} further down the script}
\PYG{k}{if} \PYG{n}{p\PYGZus{}value} \PYG{o}{\PYGZlt{}} \PYG{n}{p\PYGZus{}value\PYGZus{}threshold}\PYG{p}{:}
    \PYG{o}{.}\PYG{o}{.}\PYG{o}{.}
\end{sphinxVerbatim}
\end{sphinxadmonition}


\bigskip\hrule\bigskip



\subsubsection{⏩ Functions}
\label{\detokenize{chapters/02/02a_coding-best-practices:functions}}\begin{itemize}
\item {} 
\sphinxAtStartPar
\sphinxstylestrong{Functions, functions, functions:} Functions are reusable building blocks. Break your code into logical, reusable functions wherever possible.

\item {} 
\sphinxAtStartPar
A good rule of thumb: functions should be shorter than a page (\textasciitilde{}60 lines) and do one thing well.

\item {} 
\sphinxAtStartPar
If a code block appears more than twice, it’s probably worth turning into a function.

\end{itemize}


\bigskip\hrule\bigskip



\subsubsection{🗑️ Eliminate Duplication}
\label{\detokenize{chapters/02/02a_coding-best-practices:eliminate-duplication}}\begin{itemize}
\item {} 
\sphinxAtStartPar
\sphinxstylestrong{Within your own code:} Use functions to avoid copy\sphinxhyphen{}pasting logic.

\item {} 
\sphinxAtStartPar
\sphinxstylestrong{Beyond your code:} Don’t reinvent the wheel! Most languages offer high\sphinxhyphen{}quality libraries for common tasks. Before building from scratch, check if a well\sphinxhyphen{}maintained package already solves your problem.

\end{itemize}

\begin{figure}[htbp]
\centering
\capstart

\noindent\sphinxincludegraphics[height=380\sphinxpxdimen]{{library_functions}.png}
\caption{The right use of library functions improves not just performance, but also the clarity of your script.}\label{\detokenize{chapters/02/02a_coding-best-practices:library-functions}}\end{figure}

\begin{sphinxadmonition}{warning}{Warning:}
\sphinxAtStartPar
🧂 Be cautious when using new or experimental packages. Test imported functions before integrating them into your workflow.
\end{sphinxadmonition}


\bigskip\hrule\bigskip


\begin{figure}[htbp]
\centering
\capstart

\noindent\sphinxincludegraphics[height=400\sphinxpxdimen]{{variable_name}.png}
\caption{When in doubt, opt for descriptive variable names over short but ambiguous ones.}\label{\detokenize{chapters/02/02a_coding-best-practices:variable-names}}\end{figure}


\subsubsection{💬 Naming Matters}
\label{\detokenize{chapters/02/02a_coding-best-practices:naming-matters}}\begin{itemize}
\item {} 
\sphinxAtStartPar
\sphinxstylestrong{Use meaningful names:} Choose descriptive names for variables and functions. The broader their scope or importance, the more informative the name should be.
\begin{itemize}
\item {} 
\sphinxAtStartPar
For loop counters, short names like \sphinxcode{\sphinxupquote{i}} or \sphinxcode{\sphinxupquote{j}} are fine.

\item {} 
\sphinxAtStartPar
For important variables or data structures, avoid cryptic one\sphinxhyphen{}letter names.

\end{itemize}

\end{itemize}

\begin{sphinxadmonition}{tip}{Tip:}
\sphinxAtStartPar
\sphinxstylestrong{Tab Completion:} Most modern text editors support tab completion, so long, descriptive names don’t slow you down!
\end{sphinxadmonition}


\bigskip\hrule\bigskip



\subsubsection{📦 Dependencies}
\label{\detokenize{chapters/02/02a_coding-best-practices:dependencies}}\begin{itemize}
\item {} 
\sphinxAtStartPar
\sphinxstylestrong{List script requirements:} Include a \sphinxcode{\sphinxupquote{requirements.txt}} file (or equivalent) to specify the dependencies needed to run your code.

\item {} 
\sphinxAtStartPar
Alternatively, create a \sphinxstylestrong{Getting Started} section to your project’s \sphinxcode{\sphinxupquote{README}} file with clear setup instructions.

\end{itemize}


\bigskip\hrule\bigskip



\subsubsection{⚠️ Avoid Commenting Out Code}
\label{\detokenize{chapters/02/02a_coding-best-practices:avoid-commenting-out-code}}\begin{itemize}
\item {} 
\sphinxAtStartPar
Don’t toggle code behavior by commenting and uncommenting blocks. Instead, use \sphinxcode{\sphinxupquote{if}}/\sphinxcode{\sphinxupquote{else}} statements, configuration files, or flags for control flow.

\end{itemize}

\begin{figure}[htbp]
\centering
\capstart

\noindent\sphinxincludegraphics[height=380\sphinxpxdimen]{{bad_comments}.jpeg}
\caption{Avoid using comments to toggle code on or off—there are better ways.}\label{\detokenize{chapters/02/02a_coding-best-practices:bad-comments}}\end{figure}


\bigskip\hrule\bigskip



\subsubsection{🗂️ Archive Your Code}
\label{\detokenize{chapters/02/02a_coding-best-practices:archive-your-code}}\begin{itemize}
\item {} 
\sphinxAtStartPar
If you want to make your code citable and discoverable, consider archiving it on a DOI\sphinxhyphen{}issuing repository. \sphinxhref{https://docs.github.com/en/repositories/archiving-a-github-repository/referencing-and-citing-content}{Zenodo} integrates seamlessly with GitHub, allowing you to create a citable snapshot of your project.

\end{itemize}


\bigskip\hrule\bigskip


\sphinxAtStartPar
For further reading, you could check out the following resources:
\begin{enumerate}
\sphinxsetlistlabels{\arabic}{enumi}{enumii}{}{.}%
\item {} 
\sphinxAtStartPar
\sphinxhref{https://doi.org/10.1371/journal.pcbi.1005510}{Good enough practices in scientific computing} %
\begin{footnote}[1]\sphinxAtStartFootnote
Greg Wilson, Jennifer Bryan, Karen Cranston, Justin Kitzes, Lex Nederbragt, and Tracy K Teal. Good enough practices in scientific computing. \sphinxstyleemphasis{PLoS computational biology}, 13(6):e1005510, 2017.
%
\end{footnote}

\item {} 
\sphinxAtStartPar
\sphinxhref{https://doi.org/10.1016/j.neuroimage.2022.119623}{Open and reproducible neuroimaging: From study inception to publication} %
\begin{footnote}[2]\sphinxAtStartFootnote
Guiomar Niso, Rotem Botvinik\sphinxhyphen{}Nezer, Stefan Appelhoff, Alejandro De La Vega, Oscar Esteban, Joset A Etzel, Karolina Finc, Melanie Ganz, Rémi Gau, Yaroslav O Halchenko, and others. Open and reproducible neuroimaging: from study inception to publication. \sphinxstyleemphasis{NeuroImage}, 263:119623, 2022.
%
\end{footnote}

\item {} 
\sphinxAtStartPar
The Turing Way’s \sphinxhref{https://book.the-turing-way.org/reproducible-research/reproducible-research}{Guide for Reproducible Research}

\item {} 
\sphinxAtStartPar
Best Practices for Writing Reproducible Code, an \sphinxhref{https://utrechtuniversity.github.io/workshop-computational-reproducibility/}{online course by Utrecht University}.

\item {} 
\sphinxAtStartPar
British Ecological Society’s \sphinxhref{https://www.britishecologicalsociety.org/wp-content/uploads/2019/06/BES-Guide-Reproducible-Code-2019.pdf}{Guide for Reproducible Code}

\end{enumerate}


\bigskip\hrule\bigskip



\subsection{📑 References}
\label{\detokenize{chapters/02/02a_coding-best-practices:references}}

\bigskip\hrule\bigskip


\sphinxstepscope


\section{📘2️⃣ Using Notebooks Effectively}
\label{\detokenize{chapters/02/02b_jupyter-notebooks:using-notebooks-effectively}}\label{\detokenize{chapters/02/02b_jupyter-notebooks::doc}}
\sphinxAtStartPar
Not every research project results in a full software package. Sometimes, the core of your work is a collection of analyses or experiments that can be best expressed as a collection of code snippets, visualizations, and narrative. In these cases, preparing a full software repository might feel like overkill or a barrier to sharing. This chapter explains how Jupyter notebooks can provide a good solution to make your analysis code openly accessible.


\subsection{🚀 Jupyter Notebooks to the Rescue}
\label{\detokenize{chapters/02/02b_jupyter-notebooks:jupyter-notebooks-to-the-rescue}}
\noindent{\hspace*{\fill}\sphinxincludegraphics[width=200\sphinxpxdimen]{{jupyter-logo}.png}}

\sphinxAtStartPar
Jupyter notebooks offer a flexible, language\sphinxhyphen{}agnostic environment that blends code, rich text, and outputs in a single document. Whether you’re using Python, R, Julia, or another language, notebooks let you organize your work interactively. Combined with version control systems like Git and platforms like GitHub, notebooks provide a simple but powerful way to share reproducible research.

\sphinxAtStartPar
When you follow best coding practices (covered in the previous section), your notebooks become readable, reusable, and easier to maintain. Plus, thanks to native support on platforms like GitHub, users can preview both your code and its outputs without needing to run anything locally.


\bigskip\hrule\bigskip



\subsubsection{Notebooks Support Multiple Languages}
\label{\detokenize{chapters/02/02b_jupyter-notebooks:notebooks-support-multiple-languages}}
\sphinxAtStartPar
Jupyter is not limited to Python. It supports a wide variety of programming languages via “kernels,” which are the computational engines that execute your code. Popular kernels include Python (\sphinxhref{https://ipython.readthedocs.io/en/stable/install/kernel\_install.html}{ipython}), R (\sphinxhref{https://irkernel.github.io/installation/}{IRkernel}), Julia (\sphinxhref{https://julialang.github.io/IJulia.jl/stable/}{IJulia}), MATLAB (\sphinxhref{https://au.mathworks.com/help/cloudcenter/ug/run-matlab-in-jupyter.html}{Matlab Integration for Jupyter}) and many others. This flexibility allows you to write notebooks in the language best suited to your research domain or combine languages in the same project with multiple notebooks. (check here for an \sphinxhref{https://gist.github.com/chronitis/682c4e0d9f663e85e3d87e97cd7d1624}{exhaustive list of available kernels})

\begin{figure}[htbp]
\centering
\capstart

\noindent\sphinxincludegraphics[height=400\sphinxpxdimen]{{jlab-launcher}.png}
\caption{Whatever programming language you use, there’s a good chance it already has a Jupyter integration.}\label{\detokenize{chapters/02/02b_jupyter-notebooks:jupyter-kernels}}\end{figure}


\subsubsection{Native GitHub Rendering}
\label{\detokenize{chapters/02/02b_jupyter-notebooks:native-github-rendering}}
\sphinxAtStartPar
GitHub automatically renders Jupyter notebooks, showing both the code cells and their outputs (plots, tables, text). \sphinxhref{https://github.blog/news-insights/product-news/github-jupyter-notebooks-3/}{This} makes it easy for collaborators, reviewers, or readers to explore your work directly in their browsers without downloading or running any code. The integration boosts transparency and lowers the barrier for others to understand and build upon your results.


\subsubsection{Integration with Live Execution Environments}
\label{\detokenize{chapters/02/02b_jupyter-notebooks:integration-with-live-execution-environments}}
\sphinxAtStartPar
One of the great strengths of notebooks is how easily they can be linked to live execution platforms:
\begin{itemize}
\item {} 
\sphinxAtStartPar
\sphinxhref{https://colab.google/}{\sphinxstylestrong{Google Colab}}: Provides free cloud\sphinxhyphen{}hosted Jupyter environments with pre\sphinxhyphen{}installed libraries, GPU support, and seamless integration with \sphinxhref{https://colab.research.google.com/github/googlecolab/colabtools/blob/master/notebooks/colab-github-demo.ipynb}{GitHub}. By simply adding a \sphinxstyleemphasis{Open in Colab} badge, others can run and modify your notebook instantly, no setup required (see \sphinxhref{https://colab.research.google.com/github/googlecolab/colabtools/blob/master/notebooks/colab-github-demo.ipynb}{here} for detail).

\item {} 
\sphinxAtStartPar
\sphinxhref{https://mybinder.org/}{\sphinxstylestrong{Binder}}: Offers a way to create custom, reproducible computing environments launched directly from your GitHub repository. Binder builds a Docker image with all dependencies defined in your repo (e.g., \sphinxcode{\sphinxupquote{requirements.txt}}, \sphinxcode{\sphinxupquote{environment.yml}}), allowing users to interact with your notebooks live in their browsers. (check \sphinxhref{https://book.the-turing-way.org/communication/binder/zero-to-binder}{this step\sphinxhyphen{}by\sphinxhyphen{}step tutorial} from The Turing Way to find out how)

\end{itemize}

\sphinxAtStartPar
These services make it simple to share not just static scripts, but fully executable research notebooks.


\bigskip\hrule\bigskip



\subsubsection{Recommended directory structure}
\label{\detokenize{chapters/02/02b_jupyter-notebooks:recommended-directory-structure}}
\sphinxAtStartPar
To keep your project organized and simple, consider a clear directory layout:

\begin{sphinxVerbatim}[commandchars=\\\{\}]
──▶\PYG{+w}{ }project\PYGZus{}directory/\PYG{+w}{              }\PYG{c+c1}{\PYGZsh{} To be tracked by git}
\PYG{+w}{    }├──\PYG{+w}{ }notebooks/\PYG{+w}{                  }\PYG{c+c1}{\PYGZsh{} All Jupyter notebooks (.ipynb files)}
\PYG{+w}{    }│\PYG{+w}{   }├──\PYG{+w}{ }step\PYGZus{}1/\PYG{+w}{                 }\PYG{c+c1}{\PYGZsh{} Divide you analysis into different steps}
\PYG{+w}{    }│\PYG{+w}{   }│\PYG{+w}{   }└──\PYG{+w}{ }analysis\PYGZus{}1.ipynb
\PYG{+w}{    }│\PYG{+w}{   }├──\PYG{+w}{ }step\PYGZus{}2/
\PYG{+w}{    }│\PYG{+w}{   }│\PYG{+w}{   }└──\PYG{+w}{ }analysis\PYGZus{}2.ipynb
\PYG{+w}{    }┆\PYG{+w}{   }┆
\PYG{+w}{    }│\PYG{+w}{   }└──\PYG{+w}{ }README.md\PYG{+w}{               }\PYG{c+c1}{\PYGZsh{} Overview or instructions for notebooks}
\PYG{+w}{    }├──\PYG{+w}{ }data/\PYG{+w}{                       }\PYG{c+c1}{\PYGZsh{} Raw/processed data files}
\PYG{+w}{    }├──\PYG{+w}{ }scripts/\PYG{+w}{                    }\PYG{c+c1}{\PYGZsh{} [OPTIONAL] Standalone scripts and utility functions}
\PYG{+w}{    }├──\PYG{+w}{ }docs/\PYG{+w}{                       }\PYG{c+c1}{\PYGZsh{} [OPTIONAL] Documentation}
\PYG{+w}{    }├──\PYG{+w}{ }environment.yml\PYG{+w}{             }\PYG{c+c1}{\PYGZsh{} or requirements.txt (for dependencies)}
\PYG{+w}{    }└──\PYG{+w}{ }README.md\PYG{+w}{                   }\PYG{c+c1}{\PYGZsh{} Project overview and instructions}
\end{sphinxVerbatim}

\sphinxAtStartPar
This structure separates notebooks from scripts and data, making your project easier to manage and maintain. By adopting this layout from the beginning, you can seamlessly track your progress with Git. When your project is ready for publication, you can simply make the repository public—sharing open code (and potentially data) alongside your manuscript.


\subsubsection{Additional Points to Consider}
\label{\detokenize{chapters/02/02b_jupyter-notebooks:additional-points-to-consider}}\begin{itemize}
\item {} 
\sphinxAtStartPar
\sphinxstylestrong{Clear Narrative and Modularization}: Keep your notebooks readable by using markdown cells to explain your analysis steps, and consider modularizing complex code into external scripts or functions imported into the notebook.

\item {} 
\sphinxAtStartPar
\sphinxstylestrong{Avoid Large Outputs in Notebooks}: To keep notebooks lightweight and fast to load, avoid embedding very large data outputs or images. Instead, save large results externally and reference them.

\item {} 
\sphinxAtStartPar
\sphinxstylestrong{Use Git LFS for large files}: If your notebooks or data files are large, consider using \sphinxhref{https://docs.github.com/en/repositories/working-with-files/managing-large-files/about-git-large-file-storage}{\sphinxstylestrong{Git Large File Storage (LFS)}} to manage them efficiently within your repository.

\end{itemize}

\sphinxstepscope


\chapter{📙 Reproducible Scientific Visualizations}
\label{\detokenize{chapters/03/reproducible-visualizations:reproducible-scientific-visualizations}}\label{\detokenize{chapters/03/reproducible-visualizations::doc}}
\sphinxAtStartPar
This chapter introduces the topic of \sphinxstylestrong{programmatic scientific visualizations}, highlights their critical role in \sphinxstylestrong{open and reproducible scientific practices}, and provides practical examples and recipes that can be applied in day\sphinxhyphen{}to\sphinxhyphen{}day research workflows

\sphinxAtStartPar
\sphinxstylestrong{Key goals of this chapter:}
\begin{itemize}
\item {} 
\sphinxAtStartPar
Emphasize the importance of making visualizations reproducible and shareable.

\item {} 
\sphinxAtStartPar
Cover various resources, tools, and libraries that support reproducible scientific visualizations.

\item {} 
\sphinxAtStartPar
Provide concrete examples of reproducible code for generating common types of visualizations frequently encountered in computational neuroscience and neuroimaging research.

\end{itemize}

\sphinxstepscope


\section{📙1️⃣ Programmatic Visualizations}
\label{\detokenize{chapters/03/03a_programmatic-visualizations:programmatic-visualizations}}\label{\detokenize{chapters/03/03a_programmatic-visualizations::doc}}

\subsection{🙏 Acknowledgment}
\label{\detokenize{chapters/03/03a_programmatic-visualizations:acknowledgment}}
\sphinxAtStartPar
This chapter draws inspiration from a \sphinxhref{https://youtu.be/RTy5iVHHGO8?si=8W4eSJVzfYHGSG5x\&amp;t=90}{talk by Dr. Sid Chopra}, whose insights closely align with the themes I intended to cover here. Several key ideas are adapted, with his kind permission, from that presentation. I’m grateful to Dr. Chopra for generously allowing me to share and expand on them.


\subsection{👀 Why Visualizations Matter}
\label{\detokenize{chapters/03/03a_programmatic-visualizations:why-visualizations-matter}}\begin{quote}

\sphinxAtStartPar
If I can’t picture it, I can’t understand it.

\sphinxAtStartPar
\sphinxstyleemphasis{Albert Einstein}
\end{quote}

\sphinxAtStartPar
In scientific research, figures often outlast the text. They are what readers remember, what journalists spotlight, and sometimes even what enters public imagination and pop culture.


\subsubsection{The Impact of Visualization}
\label{\detokenize{chapters/03/03a_programmatic-visualizations:the-impact-of-visualization}}
\sphinxAtStartPar
A powerful example is the widely circulated brain connectivity figure from a psychedelic study by Petri \sphinxstyleemphasis{et al.}%
\begin{footnote}[1]\sphinxAtStartFootnote
Giovanni Petri, Paul Expert, Federico Turkheimer, Robin Carhart\sphinxhyphen{}Harris, David Nutt, Peter J Hellyer, and Francesco Vaccarino. Homological scaffolds of brain functional networks. \sphinxstyleemphasis{Journal of The Royal Society Interface}, 11(101):20140873, 2014.
%
\end{footnote} (see \hyperref[\detokenize{chapters/03/03a_programmatic-visualizations:petri}]{Fig.\@ \ref{\detokenize{chapters/03/03a_programmatic-visualizations:petri}}}). Beyond communicating the core findings, this figure captured the attention of numerous media outlets and became emblematic of the study itself.

\begin{figure}[htbp]
\centering
\capstart

\noindent\sphinxincludegraphics[height=400\sphinxpxdimen]{{petri}.jpg}
\caption{Impact of Psilocybin on brain connectivity as illustrated by by Petri \sphinxstyleemphasis{et al.}\sphinxfootnotemark[1].}\label{\detokenize{chapters/03/03a_programmatic-visualizations:petri}}\end{figure}

\sphinxAtStartPar
Its influence extended well beyond academia, appearing in TEDx talks (in \sphinxhref{https://youtu.be/8kfGaVAXeMY?si=kEiOfQ6reJ2--Spo\&amp;t=287}{Oxford}, \sphinxhref{https://youtu.be/MZIaTaNR3gk?si=xJjZjdtVMyYzMiln\&amp;t=292}{Warwick}, \sphinxhref{https://youtu.be/wmNHe0yS8RI?si=PHgPeuN4kHDcpVu7\&amp;t=538}{Aarhus}, and \sphinxhref{https://youtu.be/FyAgx\_tzh80?si=oeHtzKCjwpS4Vr9f\&amp;t=791}{Varna}) and media stories including \sphinxhref{https://www.wired.com/2014/10/magic-mushroom-brain/?\_sp=2d78efac-45d2-42e8-9a81-43392a7f4068.1748937952200}{Wired}, \sphinxhref{https://www.vox.com/future-perfect/23972716/psychedelics-meaning-science-psychedelic-mushrooms-ketamine-psilocybin-mysticism}{Vox}, \sphinxhref{https://mindmedicineaustralia.org.au/tag/magic-mushrooms/}{Mind Medicine Australia}, \sphinxhref{https://alexdreyer.medium.com/the-mesh-age-be90fdc8cd82}{Medium}, \sphinxhref{https://www.facebook.com/watch/?v=1772763436160905}{7 news}, and \sphinxhref{https://www.businessinsider.com/how-brain-changes-on-mushrooms-2014-10}{Business Insider}.

\sphinxAtStartPar
This figure became \sphinxstylestrong{the public face of the study}, an iconic example of how visualizations can amplify a paper’s reach and cultural footprint. Yet despite its popularity, the original image remains difficult to reproduce, as standards for sharing visualizations openly were not commonplace in 2014.

\sphinxAtStartPar
Fortunately, the field has evolved. Visualization science now emphasizes openness and reproducibility, supported by a growing ecosystem of tools designed to create sharable, transparent, and programmatically generated figures.


\subsection{🖥️ Code\sphinxhyphen{} vs. GUI\sphinxhyphen{}Based Visualizations}
\label{\detokenize{chapters/03/03a_programmatic-visualizations:code-vs-gui-based-visualizations}}
\sphinxAtStartPar
There are two common paradigms for creating scientific visualizations:
\begin{itemize}
\item {} 
\sphinxAtStartPar
\sphinxstylestrong{GUI\sphinxhyphen{}based visualizations}: Created through graphical user interfaces, often involving manual adjustments, screenshots, or exports. No coding is required.

\item {} 
\sphinxAtStartPar
\sphinxstylestrong{Code\sphinxhyphen{}based visualizations}: Built using scripts where the figure’s core elements are generated programmatically, with minimal manual adjustments (e.g. for annotations).

\end{itemize}

\sphinxAtStartPar
GUI\sphinxhyphen{}based workflows can be fast and intuitive, giving researchers the freedom to tweak visuals until they “look right.” However, this manual process makes it difficult to reproduce the figure, especially if the original settings are undocumented or the utilized tool is not mentioned.

\sphinxAtStartPar
In contrast, \sphinxstylestrong{code\sphinxhyphen{}based visualizations} are inherently more reproducible. Since the figure is generated from a script, it can be re\sphinxhyphen{}run by anyone, on a different machine, with updated parameters and data. This makes code\sphinxhyphen{}based approaches a foundational tool for open, transparent, and reusable science.


\subsection{✳️ Why Use Programmatic Visualizations?}
\label{\detokenize{chapters/03/03a_programmatic-visualizations:why-use-programmatic-visualizations}}
\sphinxAtStartPar
Beyond reproducibility, programmatic visualizations offer several powerful advantages:
\begin{itemize}
\item {} 
\sphinxAtStartPar
\sphinxstylestrong{Flexibility}: Seamlessly tweak styles, input data, visualization parameters, or output formats.

\item {} 
\sphinxAtStartPar
\sphinxstylestrong{Scalability}: Efficiently generate dozens or hundreds of plots in a single script; ideal for individual\sphinxhyphen{}level reports, making animations, iterative experiments, or sensitivity analyses.

\item {} 
\sphinxAtStartPar
\sphinxstylestrong{Interactivity}: Tools like Plotly, Bokeh, and Altair support interactive plots that can be embedded in notebooks or the web.

\item {} 
\sphinxAtStartPar
\sphinxstylestrong{Version control}: Code\sphinxhyphen{}generated figures can be tracked with Git alongside your analysis pipeline.

\item {} 
\sphinxAtStartPar
\sphinxstylestrong{Automation}: Integrate visualizations directly into your analysis pipelines, no manual export needed.

\item {} 
\sphinxAtStartPar
\sphinxstylestrong{Data integrity}: The connection between data and final figure is transparent and auditable, leaving no hidden/unknown steps.

\item {} 
\sphinxAtStartPar
\sphinxstylestrong{Skill reusability}: While the learning curve can be steep initially, the skills and code you develop are reusable across projects, yielding compounding returns.

\end{itemize}

\sphinxAtStartPar
These strengths make code\sphinxhyphen{}based visualization not just a stylistic preference, but a cornerstone of reproducible and transparent research, and an essential skill in any researcher’s toolkit.


\subsection{👉 What Next?}
\label{\detokenize{chapters/03/03a_programmatic-visualizations:what-next}}
\sphinxAtStartPar
In the next section, we’ll introduce a range of tools, libraries, and templates that make code\sphinxhyphen{}based visualizations easier and more powerful. We’ll explore how these tools integrate with Jupyter notebooks and neuroimaging file formats, enabling you to share interactive, reproducible, and publication\sphinxhyphen{}ready figures as part of your research outputs.


\bigskip\hrule\bigskip



\subsection{📑 References}
\label{\detokenize{chapters/03/03a_programmatic-visualizations:references}}

\bigskip\hrule\bigskip


\sphinxstepscope


\section{📙2️⃣ Visualization Tools Overview}
\label{\detokenize{chapters/03/03b_visualization-tools:visualization-tools-overview}}\label{\detokenize{chapters/03/03b_visualization-tools::doc}}
\sphinxAtStartPar
This chapter offers a broad overview of tools and libraries that support the programmatic creation of publication\sphinxhyphen{}quality figures. While I don’t consider myself an expert in scientific visualization, I’ve explored a variety of tools in my own research, and I hope this section provides a helpful starting point for others. Since my experience is primarily with Python, the tools discussed here are mostly Python\sphinxhyphen{}based. However, I’ve included links at the end of the chapter that point to similar resources for other programming environments.

\sphinxAtStartPar
Most tools are introduced only briefly, with links to more comprehensive guides. The goal is to highlight what’s possible and to help you discover tools that might suit your own research needs.

\sphinxAtStartPar
The next two sections will cover general\sphinxhyphen{}purpose and neuroimaging\sphinxhyphen{}specific visualization libraries, respectively.


\subsection{📊 General Visualization Tools}
\label{\detokenize{chapters/03/03b_visualization-tools:general-visualization-tools}}
\sphinxAtStartPar
This section briefly introduces a set of Python libraries commonly used for general\sphinxhyphen{}purpose data visualization. These tools are widely applicable across scientific disciplines and can be used to generate high\sphinxhyphen{}quality plots for publications, presentations, and exploratory analysis.


\subsubsection{Basic Plotting Libraries}
\label{\detokenize{chapters/03/03b_visualization-tools:basic-plotting-libraries}}
\sphinxAtStartPar
Two of the most widely used libraries for data visualization in Python are \sphinxhref{https://matplotlib.org/stable/plot\_types/index.html}{\sphinxstylestrong{Matplotlib}} and \sphinxhref{https://seaborn.pydata.org/examples/index.html}{\sphinxstylestrong{Seaborn}}:
\begin{itemize}
\item {} 
\sphinxAtStartPar
\sphinxhref{https://matplotlib.org/stable/plot\_types/index.html}{\sphinxstylestrong{Matplotlib}} is a powerful and flexible low\sphinxhyphen{}level library for creating plots. It offers fine\sphinxhyphen{}grained control over every aspect of a figure.

\end{itemize}

\begin{figure}[htbp]
\centering
\capstart

\noindent\sphinxincludegraphics[width=1.000\linewidth]{{matplotlib}.png}
\caption{A gallery of various plot types generated via \sphinxhref{https://matplotlib.org/stable/plot\_types/index.html}{\sphinxstylestrong{Matplotlib}}.}\label{\detokenize{chapters/03/03b_visualization-tools:matplotlib}}\end{figure}
\begin{itemize}
\item {} 
\sphinxAtStartPar
\sphinxhref{https://seaborn.pydata.org/examples/index.html}{\sphinxstylestrong{Seaborn}} is built on top of Matplotlib and provides a higher\sphinxhyphen{}level, more convenient API, particularly well\sphinxhyphen{}suited for statistical data visualization.

\end{itemize}

\begin{figure}[htbp]
\centering
\capstart

\noindent\sphinxincludegraphics[width=1.000\linewidth]{{seaborn}.png}
\caption{A gallery of various plot types generated via \sphinxhref{https://seaborn.pydata.org/examples/index.html}{\sphinxstylestrong{Seaborn}}.}\label{\detokenize{chapters/03/03b_visualization-tools:seaborn}}\end{figure}

\sphinxAtStartPar
These libraries support a wide range of common plot types—including line plots, scatter plots, bar plots, histograms, box plots, heatmaps, violin plots, and more—which you can explore in detail through the linked gallery pages.

\sphinxAtStartPar
Some key features worth highlighting:
\begin{itemize}
\item {} 
\sphinxAtStartPar
✅ \sphinxstylestrong{Multi\sphinxhyphen{}panel layouts}: You can compose complex, multi\sphinxhyphen{}panel figures entirely within Matplotlib (and Seaborn), making them suitable for direct inclusion in publications.

\end{itemize}

\begin{figure}[htbp]
\centering
\capstart

\noindent\sphinxincludegraphics[height=300\sphinxpxdimen]{{gridspec}.png}
\caption{Matplotlib’s \sphinxhref{https://matplotlib.org/stable/gallery/subplots\_axes\_and\_figures/gridspec\_multicolumn.html}{\sphinxcode{\sphinxupquote{Gridspec}}} provides a flexible way to make multiple panels in a single figure.}\label{\detokenize{chapters/03/03b_visualization-tools:gridspec}}\end{figure}
\begin{itemize}
\item {} 
\sphinxAtStartPar
✅ \sphinxstylestrong{Flexible output formats}: Export figures to various formats such as PNG, PDF, SVG, and EPS.

\item {} 
\sphinxAtStartPar
✅ \sphinxstylestrong{LaTeX rendering}: Seamless integration with LaTeX lets you render mathematical notation directly in your plots.

\end{itemize}

\begin{figure}[htbp]
\centering
\capstart

\noindent\sphinxincludegraphics[height=300\sphinxpxdimen]{{equations}.png}
\caption{Matplotlib has \sphinxhref{https://matplotlib.org/stable/gallery/text\_labels\_and\_annotations/tex\_demo.html}{built\sphinxhyphen{}in support for latex}, making it possible to integrate equations in figures.}\label{\detokenize{chapters/03/03b_visualization-tools:equations}}\end{figure}
\begin{itemize}
\item {} 
\sphinxAtStartPar
✅ \sphinxstylestrong{Integration with NumPy/Pandas}: Native support for working with common data structures.

\end{itemize}

\sphinxAtStartPar
We won’t cover how to write basic plotting scripts here, but links to tutorials and documentation are provided at the end of this chapter.


\subsubsection{Supporting Data Structures}
\label{\detokenize{chapters/03/03b_visualization-tools:supporting-data-structures}}
\sphinxAtStartPar
In neuroimaging and other data\sphinxhyphen{}intensive fields, efficiently managing and transforming large datasets is essential. Several foundational Python libraries serve as the backbone of many visualization workflows:
\begin{itemize}
\item {} 
\sphinxAtStartPar
\sphinxhref{https://numpy.org/}{\sphinxstylestrong{NumPy}}: Provides fast and efficient numerical operations on arrays and matrices.

\item {} 
\sphinxAtStartPar
\sphinxhref{https://pandas.pydata.org/}{\sphinxstylestrong{Pandas}}: Offers powerful data structures (like \sphinxcode{\sphinxupquote{DataFrame}}) for tabular data, with built\sphinxhyphen{}in methods for handling missing data, filtering, grouping, and more.

\item {} 
\sphinxAtStartPar
\sphinxhref{https://docs.xarray.dev/en/stable/}{\sphinxstylestrong{Xarray}}: Designed for N\sphinxhyphen{}dimensional labeled arrays (e.g., time × region × subject), making it especially useful for spatiotemporal datasets and more complex data formats.

\end{itemize}

\sphinxAtStartPar
Most visualization libraries can directly accept these structures as input, making them interoperable within scientific analysis pipelines. For instance, a \sphinxcode{\sphinxupquote{.csv}} file can be loaded into a \sphinxcode{\sphinxupquote{pandas.DataFrame}}, which can then be passed directly to Seaborn for visualization.


\subsubsection{Interactive Visualizations}
\label{\detokenize{chapters/03/03b_visualization-tools:interactive-visualizations}}
\sphinxAtStartPar
While static plots are great for publications, interactive visualizations are invaluable for data exploration and sharing results in dynamic formats (e.g., web dashboards, notebooks). Several libraries support building rich, interactive graphics:
\begin{itemize}
\item {} 
\sphinxAtStartPar
\sphinxhref{https://plotly.com/python/}{\sphinxstylestrong{Plotly}}: Offers intuitive syntax for interactive plots, with features like hover info, zoom, and click events.

\end{itemize}

\begin{figure}[htbp]
\centering
\capstart

\noindent\sphinxincludegraphics[width=1.000\linewidth]{{plotly}.png}
\caption{A gallery of various plot types generated via \sphinxhref{https://plotly.com/python/}{\sphinxstylestrong{Plotly}}.}\label{\detokenize{chapters/03/03b_visualization-tools:plotly}}\end{figure}

\begin{figure}[htbp]
\centering
\capstart

\noindent\sphinxincludegraphics[width=1.000\linewidth]{{plotly-interactive}.png}
\caption{\sphinxhref{https://plotly.com/python/}{\sphinxstylestrong{Plotly}} can generate interactive visualizations.}\label{\detokenize{chapters/03/03b_visualization-tools:plotly-interactive}}\end{figure}
\begin{itemize}
\item {} 
\sphinxAtStartPar
\sphinxhref{https://altair-viz.github.io/gallery/index.html\#example-gallery}{\sphinxstylestrong{Altair}}: A declarative library built on the Vega\sphinxhyphen{}Lite grammar of graphics, great for producing concise and interactive charts.

\item {} 
\sphinxAtStartPar
\sphinxhref{https://bokeh.org/}{\sphinxstylestrong{Bokeh}}: Ideal for building interactive dashboards or embedding visualizations in web applications.

\item {} 
\sphinxAtStartPar
\sphinxhref{https://panel.holoviz.org/}{\sphinxstylestrong{Panel}}: A versatile library for creating interactive apps and dashboards, supporting multiple plotting libraries (e.g., Bokeh, Plotly, Matplotlib, Altair) and well\sphinxhyphen{}suited for use in notebooks or standalone web apps.

\item {} 
\sphinxAtStartPar
\sphinxhref{https://ipywidgets.readthedocs.io/en/stable/}{\sphinxstylestrong{ipywidgets}}: Useful for adding interactivity within Jupyter notebooks, especially when combined with Matplotlib or Plotly.

\item {} 
\sphinxAtStartPar
\sphinxhref{https://dash.plotly.com/}{\sphinxstylestrong{Dash}}: A framework (built on top of Plotly) for creating full web applications with interactive graphs and controls using pure Python.

\end{itemize}

\sphinxAtStartPar
These tools are particularly useful in designing exploratory analysis, making interactive reports, and building reproducible visualizations inside notebooks.


\subsubsection{Visualizing Large Datasets with Shaders}
\label{\detokenize{chapters/03/03b_visualization-tools:visualizing-large-datasets-with-shaders}}
\sphinxAtStartPar
For very large datasets, such as scatter plots with many points, traditional plotting libraries can become unresponsive or fail to render efficiently. This is where GPU\sphinxhyphen{}accelerated, shader\sphinxhyphen{}based visualization becomes useful.

\sphinxAtStartPar
\sphinxhref{https://datashader.org/}{\sphinxstylestrong{Datashader}} (by HoloViz) is specifically designed to render huge datasets by computing aggregate representations (with GPU support). Instead of plotting individual points, it computes and shades the density of data in image space, resulting in clean, informative visualizations even with billions of points.

\begin{figure}[htbp]
\centering
\capstart

\noindent\sphinxincludegraphics[height=400\sphinxpxdimen]{{datashader}.jpg}
\caption{Datashader can be used to visualize dynamical system attractors (with 10 million points each).}\label{\detokenize{chapters/03/03b_visualization-tools:datashader}}\end{figure}


\subsubsection{Choosing Effective Colormaps}
\label{\detokenize{chapters/03/03b_visualization-tools:choosing-effective-colormaps}}
\sphinxAtStartPar
Colormaps are a fundamental component of scientific visualization, shaping how patterns and contrasts in data are perceived. Poor choices can obscure structure or introduce misleading gradients.

\sphinxAtStartPar
Key considerations:
\begin{itemize}
\item {} 
\sphinxAtStartPar
\sphinxstylestrong{Perceptual uniformity}: Changes in value should be perceived evenly across the range.

\item {} 
\sphinxAtStartPar
\sphinxstylestrong{Data type}: Use sequential colormaps for ordered data, diverging for values around a central reference (e.g., ± change), and categorical for discrete labels.

\end{itemize}

\sphinxAtStartPar
Matplotlib and Seaborn offer a variety of useful palettes; for instance, \sphinxcode{\sphinxupquote{viridis}}, \sphinxcode{\sphinxupquote{plasma}}, and \sphinxcode{\sphinxupquote{cividis}} are perceptually uniform. For a wider range of options, the \sphinxhref{https://colorcet.com/gallery.html}{\sphinxstylestrong{Colorcet}} package provides a rich set of perceptually uniform and publication\sphinxhyphen{}ready colormaps designed with clarity and aesthetics in mind.

\begin{figure}[htbp]
\centering
\capstart

\noindent\sphinxincludegraphics[width=1.000\linewidth]{{colorcet}.png}
\caption{The \sphinxhref{https://colorcet.com/gallery.html}{\sphinxstylestrong{Colorcet}} package contains a wide array of perceptually uniform colormaps to suit different use cases.}\label{\detokenize{chapters/03/03b_visualization-tools:colorcet}}\end{figure}

\sphinxAtStartPar
Using appropriate colormaps not only improves figure readability and aesthetics, but also upholds scientific integrity by avoiding unintentional distortions in the data’s visual representation.


\bigskip\hrule\bigskip



\subsection{🧠 Neuroimaging Visualization Tools}
\label{\detokenize{chapters/03/03b_visualization-tools:neuroimaging-visualization-tools}}
\sphinxAtStartPar
Having introduced a suite of general\sphinxhyphen{}purpose data visualization tools, we now turn to software packages that are specifically tailored for neuroscience and neuroimaging. These tools are designed to handle domain\sphinxhyphen{}specific data formats and support visualizations that are uniquely relevant to brain imaging research.

\begin{sphinxadmonition}{note}{Note:}
\sphinxAtStartPar
This list includes several representative tools for each visualization type, but it is by no means exhaustive. We provide links to broader, community\sphinxhyphen{}maintained lists at the end of the section and encourage readers to explore online, as the neuroimaging ecosystem is constantly evolving.
\end{sphinxadmonition}


\subsubsection{Handling Neuroimaging Data}
\label{\detokenize{chapters/03/03b_visualization-tools:handling-neuroimaging-data}}
\sphinxAtStartPar
Before visualization, neuroimaging datasets must be loaded and decoded into memory. The following Python libraries provide robust support for a variety of brain imaging file formats:
\begin{itemize}
\item {} 
\sphinxAtStartPar
\sphinxhref{https://nipy.org/nibabel/}{\sphinxstylestrong{NiBabel}}: A foundational library for reading and writing many imaging file formats such as NIfTI, CIFTI, GIFTI, and FreeSurfer surfaces/labels.

\item {} 
\sphinxAtStartPar
\sphinxhref{https://dipy.org/}{\sphinxstylestrong{DIPY}}: A powerful library for diffusion MRI analysis and tractography, including support for DICOM/NIfTI I/O.

\end{itemize}


\subsubsection{Volume Slice Rendering}
\label{\detokenize{chapters/03/03b_visualization-tools:volume-slice-rendering}}
\sphinxAtStartPar
A foundational form of neuroimaging visualization involves displaying anatomical or functional slices from 3D volumetric scans (e.g., T1\sphinxhyphen{}weighted MRI, statistical maps). These figures are ubiquitous in the literature for their clarity and ease of interpretation.

\begin{figure}[htbp]
\centering
\capstart

\noindent\sphinxincludegraphics[width=1.000\linewidth]{{vsr}.png}
\caption{An example of volume slice rendering, adapted from Bolt \sphinxstyleemphasis{et al.}\sphinxfootnotemark[1].}\label{\detokenize{chapters/03/03b_visualization-tools:vsr}}\end{figure}
%
\begin{footnotetext}[1]\sphinxAtStartFootnote
Taylor Bolt, Shiyu Wang, Jason S Nomi, Roni Setton, Benjamin P Gold, BT Yeo, J Jean Chen, Dante Picchioni, Jeff H Duyn, R Nathan Spreng, and others. Autonomic physiological coupling of the global fmri signal. \sphinxstyleemphasis{Nature Neuroscience}, pages 1–9, 2025.
%
\end{footnotetext}\ignorespaces 
\sphinxAtStartPar
📦 Python tools for generating volume slice renders:
\begin{itemize}
\item {} 
\sphinxAtStartPar
\sphinxhref{https://nipy.org/nibabel/coordinate\_systems.html}{\sphinxstylestrong{NiBabel}}

\begin{figure}[htbp]
\centering
\capstart

\noindent\sphinxincludegraphics[width=1.000\linewidth]{{nibabel}.png}
\caption{\sphinxhref{https://nipy.org/nibabel/coordinate\_systems.html}{\sphinxstylestrong{NiBabel}} provides low\sphinxhyphen{}level access to volumetric data and can be used in combination with \sphinxcode{\sphinxupquote{matplotlib.pyplot.imshow}} to generate slice visualizations..}\label{\detokenize{chapters/03/03b_visualization-tools:nibabel}}\end{figure}

\item {} 
\sphinxAtStartPar
\sphinxhref{https://nilearn.github.io/stable/auto\_examples/index.html}{\sphinxstylestrong{Nilearn}}

\begin{figure}[htbp]
\centering
\capstart

\noindent\sphinxincludegraphics[width=1.000\linewidth]{{nilearn}.png}
\caption{\sphinxhref{https://nilearn.github.io/stable/auto\_examples/index.html}{\sphinxstylestrong{Nilearn}} is a high\sphinxhyphen{}level library for statistical neuroimaging that includes built\sphinxhyphen{}in support for volume slice rendering and other visualization techniques.}\label{\detokenize{chapters/03/03b_visualization-tools:nilearn-gallery}}\end{figure}

\item {} 
\sphinxAtStartPar
\sphinxhref{https://github.com/spinicist/nanslice/tree/master}{\sphinxstylestrong{nanslice}} is a lightweight library for visualizing slices from 3D volumes.

\end{itemize}


\subsubsection{Surface\sphinxhyphen{}based Visualizations}
\label{\detokenize{chapters/03/03b_visualization-tools:surface-based-visualizations}}
\sphinxAtStartPar
Surface\sphinxhyphen{}based visualizations render cortical metrics (e.g., thickness, functional activation) on a 3D model of the cortical sheet. These projections are particularly useful for visualizing data constrained to the cortical surface, offering an intuitive, continuous view of spatial patterns that might be obscured in standard volume slice renderings.

\begin{figure}[htbp]
\centering
\capstart

\noindent\sphinxincludegraphics[width=0.600\linewidth]{{surf}.jpg}
\caption{An exemplary surface\sphinxhyphen{}based visualization depicting the principal functional gradient\sphinxfootnotemark[2], adapted from Huntenburg \sphinxstyleemphasis{et al.}\sphinxfootnotemark[3].}\label{\detokenize{chapters/03/03b_visualization-tools:surf}}\end{figure}
%
\begin{footnotetext}[2]\sphinxAtStartFootnote
Daniel S Margulies, Satrajit S Ghosh, Alexandros Goulas, Marcel Falkiewicz, Julia M Huntenburg, Georg Langs, Gleb Bezgin, Simon B Eickhoff, F Xavier Castellanos, Michael Petrides, and others. Situating the default\sphinxhyphen{}mode network along a principal gradient of macroscale cortical organization. \sphinxstyleemphasis{Proceedings of the National Academy of Sciences}, 113(44):12574–12579, 2016.
%
\end{footnotetext}\ignorespaces %
\begin{footnotetext}[3]\sphinxAtStartFootnote
Julia M Huntenburg, Pierre\sphinxhyphen{}Louis Bazin, and Daniel S Margulies. Large\sphinxhyphen{}scale gradients in human cortical organization. \sphinxstyleemphasis{Trends in cognitive sciences}, 22(1):21–31, 2018.
%
\end{footnotetext}\ignorespaces 
\sphinxAtStartPar
📦 Python tools for generating surface\sphinxhyphen{}based visualizations:
\begin{itemize}
\item {} 
\sphinxAtStartPar
\sphinxhref{https://github.com/sina-mansour/Cerebro\_Viewer}{\sphinxstylestrong{Cerebro Brain Viewer}}

\begin{figure}[htbp]
\centering
\capstart

\noindent\sphinxincludegraphics[width=1.000\linewidth]{{cerebro_surf}.png}
\caption{A set of surface\sphinxhyphen{}based visualizations from \sphinxhref{https://github.com/sina-mansour/Cerebro\_Viewer}{\sphinxstylestrong{Cerebro Brain Viewer}}.}\label{\detokenize{chapters/03/03b_visualization-tools:cerebro-surf}}\end{figure}

\item {} 
\sphinxAtStartPar
\sphinxhref{https://github.com/feilong/brainplotlib}{\sphinxstylestrong{Brainplotlib}}

\begin{figure}[htbp]
\centering
\capstart

\noindent\sphinxincludegraphics[width=0.500\linewidth]{{brainplotlib}.png}
\caption{An example surface\sphinxhyphen{}based visualizations from \sphinxhref{https://github.com/feilong/brainplotlib}{\sphinxstylestrong{Brainplotlib}}.}\label{\detokenize{chapters/03/03b_visualization-tools:brainplotlib}}\end{figure}

\item {} 
\sphinxAtStartPar
\sphinxhref{https://pysurfer.github.io/auto\_examples/index.html}{\sphinxstylestrong{PySurfer}}

\begin{figure}[htbp]
\centering
\capstart

\noindent\sphinxincludegraphics[width=0.400\linewidth]{{pysurfer}.png}
\caption{An example surface\sphinxhyphen{}based visualizations from \sphinxhref{https://pysurfer.github.io/auto\_examples/index.html}{\sphinxstylestrong{PySurfer}}.}\label{\detokenize{chapters/03/03b_visualization-tools:pysurfer}}\end{figure}

\item {} 
\sphinxAtStartPar
\sphinxhref{https://www.nitrc.org/plugins/mwiki/index.php/surfice:MainPage}{\sphinxstylestrong{SurfIce}}

\begin{figure}[htbp]
\centering
\capstart

\noindent\sphinxincludegraphics[width=0.400\linewidth]{{surfice}.jpg}
\caption{An example surface\sphinxhyphen{}based visualizations from \sphinxhref{https://www.nitrc.org/plugins/mwiki/index.php/surfice:MainPage}{\sphinxstylestrong{SurfIce}}.}\label{\detokenize{chapters/03/03b_visualization-tools:surfice}}\end{figure}

\end{itemize}


\subsubsection{Volume\sphinxhyphen{}to\sphinxhyphen{}Surface Transformation}
\label{\detokenize{chapters/03/03b_visualization-tools:volume-to-surface-transformation}}
\sphinxAtStartPar
When data originates in volumetric space (e.g., atlas\sphinxhyphen{}based ROIs, or a brain mask), it can be useful to project it onto the cortical surface for clearer spatial interpretation.

\begin{figure}[htbp]
\centering
\capstart

\noindent\sphinxincludegraphics[width=0.600\linewidth]{{yba}.png}
\caption{An exemplary ROI to surface transformation of the Yale Brain Atlas, adapted from McGrath \sphinxstyleemphasis{et al.}\sphinxfootnotemark[4].}\label{\detokenize{chapters/03/03b_visualization-tools:yba}}\end{figure}
%
\begin{footnotetext}[4]\sphinxAtStartFootnote
Hari McGrath, Hitten P Zaveri, Evan Collins, Tamara Jafar, Omar Chishti, Sami Obaid, Alexander Ksendzovsky, Kun Wu, Xenophon Papademetris, and Dennis D Spencer. High\sphinxhyphen{}resolution cortical parcellation based on conserved brain landmarks for localization of multimodal data to the nearest centimeter. \sphinxstyleemphasis{Scientific reports}, 12(1):18778, 2022.
%
\end{footnotetext}\ignorespaces 
\sphinxAtStartPar
📦 Python tools for generating volume to surface transformations:
\begin{itemize}
\item {} 
\sphinxAtStartPar
\sphinxhref{https://github.com/sina-mansour/Cerebro\_Viewer}{\sphinxstylestrong{Cerebro Brain Viewer}}

\begin{figure}[htbp]
\centering
\capstart

\noindent\sphinxincludegraphics[width=0.400\linewidth]{{cerebro_volsurf}.png}
\caption{An exemplary surface transformation of subcortical structures from \sphinxhref{https://github.com/sina-mansour/Cerebro\_Viewer}{\sphinxstylestrong{Cerebro Brain Viewer}}.}\label{\detokenize{chapters/03/03b_visualization-tools:cerebro-volsurf}}\end{figure}

\item {} 
\sphinxAtStartPar
\sphinxhref{https://www.nitrc.org/plugins/mwiki/index.php/surfice:MainPage}{\sphinxstylestrong{SurfIce}}

\begin{figure}[htbp]
\centering
\capstart

\noindent\sphinxincludegraphics[width=0.400\linewidth]{{surfice_volsurf}.jpg}
\caption{An example surface\sphinxhyphen{}based visualizations of the AICHA template from \sphinxhref{https://www.nitrc.org/plugins/mwiki/index.php?title=File:Surfice:Aicha.jpg}{\sphinxstylestrong{SurfIce}}.}\label{\detokenize{chapters/03/03b_visualization-tools:surfice-volsurf}}\end{figure}

\end{itemize}


\subsubsection{Tractography Visualization}
\label{\detokenize{chapters/03/03b_visualization-tools:tractography-visualization}}
\sphinxAtStartPar
Visualizing white matter fiber bundles from diffusion MRI is a key part of tractography\sphinxhyphen{}based studies. These plots often overlay streamlines on anatomical backdrops or 3D renderings of the brain.

\begin{figure}[htbp]
\centering
\capstart

\noindent\sphinxincludegraphics[width=1.000\linewidth]{{tract}.png}
\caption{An exemplary tractography visualization of the white\sphinxhyphen{}matter bundles, adapted from Cox \sphinxstyleemphasis{et al.}\sphinxfootnotemark[5].}\label{\detokenize{chapters/03/03b_visualization-tools:bundle}}\end{figure}
%
\begin{footnotetext}[5]\sphinxAtStartFootnote
Simon R Cox, Stuart J Ritchie, Elliot M Tucker\sphinxhyphen{}Drob, David C Liewald, Saskia P Hagenaars, Gail Davies, Joanna M Wardlaw, Catharine R Gale, Mark E Bastin, and Ian J Deary. Ageing and brain white matter structure in 3,513 uk biobank participants. \sphinxstyleemphasis{Nature communications}, 7(1):13629, 2016.
%
\end{footnotetext}\ignorespaces 
\sphinxAtStartPar
📦 Python tools for tractography visualization:
\begin{itemize}
\item {} 
\sphinxAtStartPar
\sphinxhref{https://dipy.org/}{\sphinxstylestrong{DIPY}}

\begin{figure}[htbp]
\centering
\capstart

\noindent\sphinxincludegraphics[width=0.800\linewidth]{{dipy_tract}.png}
\caption{An example visualization of tractography streamlines along the corpus callosum via \sphinxhref{https://docs.dipy.org/stable/examples\_built/visualization/viz\_roi\_contour.html\#sphx-glr-examples-built-visualization-viz-roi-contour-py}{\sphinxstylestrong{DIPY}}.}\label{\detokenize{chapters/03/03b_visualization-tools:dipy-tract}}\end{figure}

\item {} 
\sphinxAtStartPar
\sphinxhref{https://www.nitrc.org/plugins/mwiki/index.php/surfice:MainPage}{\sphinxstylestrong{SurfIce}}

\begin{figure}[htbp]
\centering
\capstart

\noindent\sphinxincludegraphics[width=0.600\linewidth]{{surfice_tract}.jpg}
\caption{An example tractography visualizations from \sphinxhref{https://www.nitrc.org/plugins/mwiki/index.php?title=File:Surfice:Fibers.jpg}{\sphinxstylestrong{SurfIce}}.}\label{\detokenize{chapters/03/03b_visualization-tools:surfice-tract}}\end{figure}

\end{itemize}


\subsubsection{Brain Network Visualizations}
\label{\detokenize{chapters/03/03b_visualization-tools:brain-network-visualizations}}
\sphinxAtStartPar
Connectivity\sphinxhyphen{}based neuroscience  research often utilizes dedicated network visualizations such as adjacency matrices (heatmaps), chord diagrams, or 3D brain network plots.

\begin{figure}[htbp]
\centering
\capstart

\noindent\sphinxincludegraphics[width=1.000\linewidth]{{network}.png}
\caption{Different visualizations of brain connectivity information via (A) heatmaps, adapted from Zamani Esfahlani \sphinxstyleemphasis{et al.}\sphinxfootnotemark[6], (B) chord diagrams, adapted from Klauser \sphinxstyleemphasis{et al.}\sphinxfootnotemark[7], and (C) network plots, adapted from Seguin \sphinxstyleemphasis{et al.}\sphinxfootnotemark[8].}\label{\detokenize{chapters/03/03b_visualization-tools:network}}\end{figure}
%
\begin{footnotetext}[6]\sphinxAtStartFootnote
Farnaz Zamani Esfahlani, Joshua Faskowitz, Jonah Slack, Bratislav Mišić, and Richard F Betzel. Local structure\sphinxhyphen{}function relationships in human brain networks across the lifespan. \sphinxstyleemphasis{Nature communications}, 13(1):2053, 2022.
%
\end{footnotetext}\ignorespaces %
\begin{footnotetext}[7]\sphinxAtStartFootnote
Paul Klauser, Vanessa L Cropley, Philipp S Baumann, Jinglei Lv, Pascal Steullet, Daniella Dwir, Yasser Alemán\sphinxhyphen{}Gómez, Meritxell Bach Cuadra, Michel Cuenod, Kim Q Do, and others. White matter alterations between brain network hubs underlie processing speed impairment in patients with schizophrenia. \sphinxstyleemphasis{Schizophrenia Bulletin Open}, 2(1):sgab033, 2021.
%
\end{footnotetext}\ignorespaces %
\begin{footnotetext}[8]\sphinxAtStartFootnote
Caio Seguin, Olaf Sporns, and Andrew Zalesky. Brain network communication: concepts, models and applications. \sphinxstyleemphasis{Nature reviews neuroscience}, 24(9):557–574, 2023.
%
\end{footnotetext}\ignorespaces 
\sphinxAtStartPar
📦 Python tools for brain connectivity visualization:

\sphinxAtStartPar
The following software packages can be used to produce these maps:
\begin{itemize}
\item {} 
\sphinxAtStartPar
Heatmaps:
\begin{itemize}
\item {} 
\sphinxAtStartPar
General purpose libraries like Matplotlib, Seaborn, and Plotly can be used to programmatically generate connectivity matrix heatmaps.

\item {} 
\sphinxAtStartPar
\sphinxhref{https://nilearn.github.io/stable/auto\_examples/03\_connectivity/plot\_probabilistic\_atlas\_extraction.html\#sphx-glr-auto-examples-03-connectivity-plot-probabilistic-atlas-extraction-py}{\sphinxstylestrong{Nilearn}} contains functions to automate this process.

\end{itemize}

\item {} 
\sphinxAtStartPar
Chord diagrams:
\begin{itemize}
\item {} 
\sphinxAtStartPar
Specific packages such as \sphinxhref{https://github.com/ponnhide/pyCircos}{\sphinxstylestrong{pyCircos}}, \sphinxhref{https://github.com/shahinrostami/chord}{\sphinxstylestrong{Chord}}, and \sphinxhref{https://github.com/pke1029/open-chord}{\sphinxstylestrong{OpenChord}} were specifically built to make chord diagrams.

\end{itemize}

\begin{figure}[htbp]
\centering
\capstart

\noindent\sphinxincludegraphics[width=0.400\linewidth]{{pycircos}.png}
\caption{Example chord diagram made by \sphinxhref{https://github.com/ponnhide/pyCircos}{\sphinxstylestrong{pyCircos}}.}\label{\detokenize{chapters/03/03b_visualization-tools:pycircos}}\end{figure}
\begin{itemize}
\item {} 
\sphinxAtStartPar
The \sphinxhref{https://mne.tools/stable/index.html}{\sphinxstylestrong{MNE}} tools library also has dedicated a section on chord diagrams for connectivity.

\end{itemize}

\begin{figure}[htbp]
\centering
\capstart

\noindent\sphinxincludegraphics[width=0.400\linewidth]{{mne_chord}.png}
\caption{An example chord diagram visualizations from \sphinxhref{https://mne.tools/mne-connectivity/stable/auto\_examples/mne\_inverse\_label\_connectivity.html\#make-a-connectivity-plot}{\sphinxstylestrong{MNE Connectivity}}.}\label{\detokenize{chapters/03/03b_visualization-tools:mne-chord}}\end{figure}

\item {} 
\sphinxAtStartPar
Brain Network visualizations:
\begin{itemize}
\item {} 
\sphinxAtStartPar
\sphinxhref{https://github.com/sina-mansour/Cerebro\_Viewer}{\sphinxstylestrong{Cerebro Brain Viewer}}

\end{itemize}

\begin{figure}[htbp]
\centering
\capstart

\noindent\sphinxincludegraphics[width=0.800\linewidth]{{cerebro_network}.png}
\caption{A 3D brain network visualization from \sphinxhref{https://github.com/sina-mansour/Cerebro\_Viewer}{\sphinxstylestrong{Cerebro Brain Viewer}}.}\label{\detokenize{chapters/03/03b_visualization-tools:cerebro-network}}\end{figure}
\begin{itemize}
\item {} 
\sphinxAtStartPar
\sphinxhref{https://www.nitrc.org/plugins/mwiki/index.php/surfice:MainPage}{\sphinxstylestrong{SurfIce}}

\end{itemize}

\begin{figure}[htbp]
\centering
\capstart

\noindent\sphinxincludegraphics[width=0.600\linewidth]{{surfice_network}.png}
\caption{An example brain network visualizations from \sphinxhref{https://www.nitrc.org/plugins/mwiki/index.php?title=File:Surfice:Nodes.png}{\sphinxstylestrong{SurfIce}}.}\label{\detokenize{chapters/03/03b_visualization-tools:surfice-network}}\end{figure}

\end{itemize}


\subsubsection{More Complex Visualizations}
\label{\detokenize{chapters/03/03b_visualization-tools:more-complex-visualizations}}
\sphinxAtStartPar
While beyond the scope of this 30\sphinxhyphen{}minute educational session, it should be noted that you could leverage full\sphinxhyphen{}fledged 3D rendering engines to create cinematic high\sphinxhyphen{}quality visuals and animations from neuroimaging data. For instance, you could use Python to drive visualizations in Blender, a powerful open\sphinxhyphen{}source graphics suite.

\sphinxAtStartPar
🎞️ For example, here is an animation created using a Python\sphinxhyphen{}Blender script:



\sphinxAtStartPar
The script to reproduce this figure is available \sphinxhref{https://gist.github.com/sina-mansour/14c3ecff56b51d2ba5a3d0b2da7deec9}{here}, provided for individuals interested in further diving down this rabbit hole! 🐇🕳️


\subsection{💠 Supplemental Guides and Resources}
\label{\detokenize{chapters/03/03b_visualization-tools:supplemental-guides-and-resources}}
\sphinxAtStartPar
The neuroimaging community is continuously developing and curating exhaustive lists of visualization tools across multiple programming languages. As promised, below are a few recommended resources to explore further. These include methodological papers, curated galleries, and practical tools to help you go beyond the examples provided in this chapter.


\subsubsection{📚 Key Publications}
\label{\detokenize{chapters/03/03b_visualization-tools:key-publications}}\begin{itemize}
\item {} 
\sphinxAtStartPar
Pernet and Madan%
\begin{footnote}[9]\sphinxAtStartFootnote
Cyril R Pernet and Christopher R Madan. Data visualization for inference in tomographic brain imaging. \sphinxstyleemphasis{European Journal of Neuroscience}, 2019.
%
\end{footnote}’s “Data visualization for inference in tomographic brain imaging” provides a structured guide to visualization choices for brain imaging, with helpful discussions on colormap selection and interpretability.

\item {} 
\sphinxAtStartPar
Chopra \sphinxstyleemphasis{et al.}%
\begin{footnote}[10]\sphinxAtStartFootnote
Sidhant Chopra, Loïc Labache, Elvisha Dhamala, Edwina R Orchard, and Avram Holmes. A practical guide for generating reproducible and programmatic neuroimaging visualizations. \sphinxstyleemphasis{Aperture Neuro}, 3:1–20, 2023.
%
\end{footnote}’s “A Practical Guide for Generating Reproducible and Programmatic Neuroimaging Visualizations” presents a cross\sphinxhyphen{}language (R, Python, MATLAB) survey of tools with a focus on reproducibility and best practices.

\item {} 
\sphinxAtStartPar
Chamberland \sphinxstyleemphasis{et al.}%
\begin{footnote}[11]\sphinxAtStartFootnote
Maxime Chamberland, Charles Poirier, Tom Hendriks, Dmitri Shastin, Anna Vilanova, and Alexander Leemans. Tractography visualization. In \sphinxstyleemphasis{Handbook of Diffusion MR Tractography}, pages 381–393. Elsevier, 2025.
%
\end{footnote}’s “Tractography visualization” (Chapter in the \sphinxhref{https://www.sciencedirect.com/book/9780128188941/handbook-of-diffusion-mr-tractography}{Handbook of Diffusion MR Tractography}) offers an in\sphinxhyphen{}depth look at diffusion imaging and fiber tracking visualization tools.

\end{itemize}


\subsubsection{🛠️ Tools and Curated Resources}
\label{\detokenize{chapters/03/03b_visualization-tools:tools-and-curated-resources}}\begin{itemize}
\item {} 
\sphinxAtStartPar
\sphinxhref{https://python-graph-gallery.com/}{\sphinxstylestrong{Python Graph Gallery}}: A comprehensive collection of general\sphinxhyphen{}purpose visualization examples built with matplotlib, seaborn, plotly, and more.

\item {} 
\sphinxAtStartPar
\sphinxhref{https://www.datacamp.com/cheat-sheet/data-viz-cheat-sheet}{\sphinxstylestrong{DataCamp’s Data Visualization Cheat Sheet}}: A tutorial on most common general purpose visualizations and where to use them.

\item {} 
\sphinxAtStartPar
\sphinxhref{https://sidchop.shinyapps.io/braincode/}{\sphinxstylestrong{BrainCode}} A code template generator for programmatic brain visualizations in R and Python. Great for learning syntax.

\item {} 
\sphinxAtStartPar
\sphinxhref{https://neurohackademy.org/course/data-visualization-in-python/}{\sphinxstylestrong{NeuroHackAcademy’s Data Visualization in Python}} Lecture is also worth checking out.

\end{itemize}


\bigskip\hrule\bigskip



\subsection{📑 References}
\label{\detokenize{chapters/03/03b_visualization-tools:references}}

\bigskip\hrule\bigskip


\sphinxstepscope


\section{📙3️⃣ Practical Visualization Examples}
\label{\detokenize{chapters/03/03c_visualization-examples:practical-visualization-examples}}\label{\detokenize{chapters/03/03c_visualization-examples::doc}}
\sphinxAtStartPar
This notebook presents hands\sphinxhyphen{}on examples that build on the concepts introduced in earlier chapters. It is intended as a practical reference, offering reusable code snippets and templates that you can adapt for your own projects. Use this resource to deepen your understanding and to develop reproducible, high\sphinxhyphen{}quality visualizations.




\subsection{Notebook Preparations}
\label{\detokenize{chapters/03/03c_visualization-examples:notebook-preparations}}
\sphinxAtStartPar
Before diving into the visualization scripts, run the following (collapsed) setup cells to:
\begin{itemize}
\item {} 
\sphinxAtStartPar
Connect to Google Colab (if applicable)

\item {} 
\sphinxAtStartPar
Import all necessary packages

\item {} 
\sphinxAtStartPar
Configure the required directory structure

\item {} 
\sphinxAtStartPar
Learn about the data used in these examples

\end{itemize}

\sphinxAtStartPar
Once these steps are complete, you’ll be ready to begin running the visualization workflows.




\subsubsection{Google Colab}
\label{\detokenize{chapters/03/03c_visualization-examples:google-colab}}
\sphinxAtStartPar
This chapter is designed to be fully interactive and can be run directly in a Google Colab environment. This allows you to experiment with the provided scripts, modify parameters, and explore how different choices affect the resulting visualizations.

\sphinxAtStartPar
To open this notebook in Colab, use the link below:



\sphinxAtStartPar
If you’re running this notebook in Colab, be sure to execute the cell below to install all required dependencies.

\begin{sphinxuseclass}{cell}
\begin{sphinxuseclass}{tag_skip-execution}\begin{sphinxVerbatimInput}

\begin{sphinxuseclass}{cell_input}
\begin{sphinxVerbatim}[commandchars=\\\{\}]
\PYG{o}{\PYGZpc{}\PYGZpc{}}\PYG{k}{bash}

\PYGZsh{} clone the repository
git clone https://github.com/sina\PYGZhy{}mansour/ohbm2025\PYGZhy{}reproducible\PYGZhy{}research.git

\PYGZsh{} install requirements
cd \PYGZdq{}ohbm2025\PYGZhy{}reproducible\PYGZhy{}research\PYGZdq{}
pip install \PYGZhy{}r requirements.txt
\end{sphinxVerbatim}

\end{sphinxuseclass}\end{sphinxVerbatimInput}

\end{sphinxuseclass}
\end{sphinxuseclass}
\begin{sphinxuseclass}{cell}
\begin{sphinxuseclass}{tag_skip-execution}\begin{sphinxVerbatimInput}

\begin{sphinxuseclass}{cell_input}
\begin{sphinxVerbatim}[commandchars=\\\{\}]
\PYG{k+kn}{import}\PYG{+w}{ }\PYG{n+nn}{os}

\PYG{c+c1}{\PYGZsh{} change to notebook directory}
\PYG{n}{os}\PYG{o}{.}\PYG{n}{chdir}\PYG{p}{(}\PYG{l+s+s2}{\PYGZdq{}}\PYG{l+s+s2}{ohbm2025\PYGZhy{}reproducible\PYGZhy{}research/ohbm2025\PYGZhy{}reproducible\PYGZhy{}research/chapters/03/}\PYG{l+s+s2}{\PYGZdq{}}\PYG{p}{)}
\end{sphinxVerbatim}

\end{sphinxuseclass}\end{sphinxVerbatimInput}

\end{sphinxuseclass}
\end{sphinxuseclass}

\subsubsection{Package Imports}
\label{\detokenize{chapters/03/03c_visualization-examples:package-imports}}
\sphinxAtStartPar
The cell below imports all the packages required to run this notebook. It assumes that the necessary dependencies listed in \sphinxcode{\sphinxupquote{requirements.txt}} have already been installed.

\begin{sphinxuseclass}{cell}\begin{sphinxVerbatimInput}

\begin{sphinxuseclass}{cell_input}
\begin{sphinxVerbatim}[commandchars=\\\{\}]
\PYG{k+kn}{import}\PYG{+w}{ }\PYG{n+nn}{os}
\PYG{k+kn}{import}\PYG{+w}{ }\PYG{n+nn}{sys}
\PYG{k+kn}{import}\PYG{+w}{ }\PYG{n+nn}{numpy}\PYG{+w}{ }\PYG{k}{as}\PYG{+w}{ }\PYG{n+nn}{np}
\PYG{k+kn}{import}\PYG{+w}{ }\PYG{n+nn}{matplotlib}\PYG{+w}{ }\PYG{k}{as}\PYG{+w}{ }\PYG{n+nn}{mpl}
\PYG{k+kn}{import}\PYG{+w}{ }\PYG{n+nn}{matplotlib}\PYG{n+nn}{.}\PYG{n+nn}{pyplot}\PYG{+w}{ }\PYG{k}{as}\PYG{+w}{ }\PYG{n+nn}{plt}
\PYG{k+kn}{from}\PYG{+w}{ }\PYG{n+nn}{scipy}\PYG{+w}{ }\PYG{k+kn}{import} \PYG{n}{sparse}
\PYG{k+kn}{import}\PYG{+w}{ }\PYG{n+nn}{colorcet}\PYG{+w}{ }\PYG{k}{as}\PYG{+w}{ }\PYG{n+nn}{cc}
\PYG{k+kn}{import}\PYG{+w}{ }\PYG{n+nn}{contextlib}
\PYG{k+kn}{from}\PYG{+w}{ }\PYG{n+nn}{io}\PYG{+w}{ }\PYG{k+kn}{import} \PYG{n}{StringIO}
\PYG{k+kn}{import}\PYG{+w}{ }\PYG{n+nn}{nibabel}\PYG{+w}{ }\PYG{k}{as}\PYG{+w}{ }\PYG{n+nn}{nib}
\PYG{k+kn}{import}\PYG{+w}{ }\PYG{n+nn}{json}

\PYG{c+c1}{\PYGZsh{} Cerebro brain viewer}
\PYG{k+kn}{from}\PYG{+w}{ }\PYG{n+nn}{cerebro}\PYG{+w}{ }\PYG{k+kn}{import} \PYG{n}{cerebro\PYGZus{}brain\PYGZus{}utils} \PYG{k}{as} \PYG{n}{cbu}
\PYG{k+kn}{from}\PYG{+w}{ }\PYG{n+nn}{cerebro}\PYG{+w}{ }\PYG{k+kn}{import} \PYG{n}{cerebro\PYGZus{}brain\PYGZus{}viewer} \PYG{k}{as} \PYG{n}{cbv}
\end{sphinxVerbatim}

\end{sphinxuseclass}\end{sphinxVerbatimInput}

\end{sphinxuseclass}

\subsubsection{Directory Setup}
\label{\detokenize{chapters/03/03c_visualization-examples:directory-setup}}
\sphinxAtStartPar
Next, let’s set up the directory structure that will be used throughout this tutorial. In this case, we’ll simply change into the appropriate directory, as the required data is already included as part of this repository.

\begin{sphinxuseclass}{cell}\begin{sphinxVerbatimInput}

\begin{sphinxuseclass}{cell_input}
\begin{sphinxVerbatim}[commandchars=\\\{\}]
\PYG{c+c1}{\PYGZsh{} Initialize the working directory}
\PYG{n}{working\PYGZus{}directory} \PYG{o}{=} \PYG{n}{os}\PYG{o}{.}\PYG{n}{getcwd}\PYG{p}{(}\PYG{p}{)}
\PYG{c+c1}{\PYGZsh{} create a flag to indicate whether the directory setup is complete}
\PYG{k}{if} \PYG{l+s+s1}{\PYGZsq{}}\PYG{l+s+s1}{directory\PYGZus{}setup\PYGZus{}complete}\PYG{l+s+s1}{\PYGZsq{}} \PYG{o+ow}{not} \PYG{o+ow}{in} \PYG{n+nb}{globals}\PYG{p}{(}\PYG{p}{)}\PYG{p}{:}
    \PYG{n}{directory\PYGZus{}setup\PYGZus{}complete} \PYG{o}{=} \PYG{k+kc}{False}
\PYG{c+c1}{\PYGZsh{} only proceed if the directory setup is not complete}
\PYG{k}{if} \PYG{o+ow}{not} \PYG{n}{directory\PYGZus{}setup\PYGZus{}complete}\PYG{p}{:}
    \PYG{c+c1}{\PYGZsh{} change the working directory to the home directory of the repository}
    \PYG{n}{os}\PYG{o}{.}\PYG{n}{chdir}\PYG{p}{(}\PYG{l+s+s2}{\PYGZdq{}}\PYG{l+s+s2}{../../..}\PYG{l+s+s2}{\PYGZdq{}}\PYG{p}{)}
    \PYG{c+c1}{\PYGZsh{} print the current working directory to make sure it is correct}
    \PYG{n}{working\PYGZus{}directory} \PYG{o}{=} \PYG{n}{os}\PYG{o}{.}\PYG{n}{getcwd}\PYG{p}{(}\PYG{p}{)}
    \PYG{n+nb}{print}\PYG{p}{(}\PYG{l+s+sa}{f}\PYG{l+s+s2}{\PYGZdq{}}\PYG{l+s+s2}{Current working directory: }\PYG{l+s+si}{\PYGZob{}}\PYG{n}{working\PYGZus{}directory}\PYG{l+s+si}{\PYGZcb{}}\PYG{l+s+s2}{\PYGZdq{}}\PYG{p}{)}
    \PYG{c+c1}{\PYGZsh{} set the flag to indicate that the directory setup is complete}
    \PYG{n}{directory\PYGZus{}setup\PYGZus{}complete} \PYG{o}{=} \PYG{k+kc}{True}
\PYG{k}{else}\PYG{p}{:}
    \PYG{c+c1}{\PYGZsh{} print a message indicating that the directory setup is already complete}
    \PYG{n+nb}{print}\PYG{p}{(}\PYG{l+s+s2}{\PYGZdq{}}\PYG{l+s+s2}{Directory setup is already complete. No changes made.}\PYG{l+s+s2}{\PYGZdq{}}\PYG{p}{)}
    \PYG{n+nb}{print}\PYG{p}{(}\PYG{l+s+sa}{f}\PYG{l+s+s2}{\PYGZdq{}}\PYG{l+s+s2}{Current working directory: }\PYG{l+s+si}{\PYGZob{}}\PYG{n}{working\PYGZus{}directory}\PYG{l+s+si}{\PYGZcb{}}\PYG{l+s+s2}{\PYGZdq{}}\PYG{p}{)}
\end{sphinxVerbatim}

\end{sphinxuseclass}\end{sphinxVerbatimInput}
\begin{sphinxVerbatimOutput}

\begin{sphinxuseclass}{cell_output}
\begin{sphinxVerbatim}[commandchars=\\\{\}]
Current working directory: /mnt/local\PYGZus{}storage/Research/Codes/jupyterbooks/ohbm2025\PYGZhy{}reproducible\PYGZhy{}research
\end{sphinxVerbatim}

\end{sphinxuseclass}\end{sphinxVerbatimOutput}

\end{sphinxuseclass}

\subsubsection{Tutorial Data}
\label{\detokenize{chapters/03/03c_visualization-examples:tutorial-data}}
\sphinxAtStartPar
A minimal set of neuroimaging files has been prepared to support the execution of the examples in this notebook.

\sphinxAtStartPar
\sphinxstylestrong{⚠️ Note}: This curated data is provided exclusively for educational purposes as part of this Jupyter Book tutorial.

\sphinxAtStartPar
For access to complete datasets or for use beyond this tutorial, please refer to the original data sources listed below.


\subsubsection{Data Sources}
\label{\detokenize{chapters/03/03c_visualization-examples:data-sources}}
\sphinxAtStartPar
The neuroimaging data used in this tutorial are derived from the following publicly available resources:
\begin{itemize}
\item {} 
\sphinxAtStartPar
\sphinxstylestrong{Human Connectome Project}’s \sphinxhref{https://www.humanconnectome.org/study/hcp-young-adult/article/s1200-group-average-data-release}{Group Average Adult Template} (see Glasser \sphinxstyleemphasis{et al.}%
\begin{footnote}[1]\sphinxAtStartFootnote
Matthew F Glasser, Stamatios N Sotiropoulos, J Anthony Wilson, Timothy S Coalson, Bruce Fischl, Jesper L Andersson, Junqian Xu, Saad Jbabdi, Matthew Webster, Jonathan R Polimeni, and others. The minimal preprocessing pipelines for the human connectome project. \sphinxstyleemphasis{Neuroimage}, 80:105–124, 2013.
%
\end{footnote}, Van Essen \sphinxstyleemphasis{et al.}%
\begin{footnote}[2]\sphinxAtStartFootnote
David C Van Essen, Kamil Ugurbil, Edward Auerbach, Deanna Barch, Timothy EJ Behrens, Richard Bucholz, Acer Chang, Liyong Chen, Maurizio Corbetta, Sandra W Curtiss, and others. The human connectome project: a data acquisition perspective. \sphinxstyleemphasis{Neuroimage}, 62(4):2222–2231, 2012.
%
\end{footnote}, and Marcus \sphinxstyleemphasis{et al.}%
\begin{footnote}[3]\sphinxAtStartFootnote
Daniel S Marcus, Michael P Harms, Abraham Z Snyder, Mark Jenkinson, J Anthony Wilson, Matthew F Glasser, Deanna M Barch, Kevin A Archie, Gregory C Burgess, Mohana Ramaratnam, and others. Human connectome project informatics: quality control, database services, and data visualization. \sphinxstyleemphasis{Neuroimage}, 80:202–219, 2013.
%
\end{footnote})

\item {} 
\sphinxAtStartPar
The Glasser Cortical brain atlas (see Glasser \sphinxstyleemphasis{et al.}%
\begin{footnote}[4]\sphinxAtStartFootnote
Matthew F Glasser, Timothy S Coalson, Emma C Robinson, Carl D Hacker, John Harwell, Essa Yacoub, Kamil Ugurbil, Jesper Andersson, Christian F Beckmann, Mark Jenkinson, and others. A multi\sphinxhyphen{}modal parcellation of human cerebral cortex. \sphinxstyleemphasis{Nature}, 536(7615):171–178, 2016.
%
\end{footnote})

\item {} 
\sphinxAtStartPar
Tractography data from the \sphinxhref{https://github.com/SlicerDMRI/ORG-Atlases?tab=readme-ov-file}{ORG fiber clustering atlas} (see Zhang \sphinxstyleemphasis{et al.}%
\begin{footnote}[5]\sphinxAtStartFootnote
Fan Zhang, Ye Wu, Isaiah Norton, Laura Rigolo, Yogesh Rathi, Nikos Makris, and Lauren J O’Donnell. An anatomically curated fiber clustering white matter atlas for consistent white matter tract parcellation across the lifespan. \sphinxstyleemphasis{Neuroimage}, 179:429–447, 2018.
%
\end{footnote})

\item {} 
\sphinxAtStartPar
Functional connectivity data from Mansour \sphinxstyleemphasis{et al.}%
\begin{footnote}[6]\sphinxAtStartFootnote
Sina Mansour, Ye Tian, BT Thomas Yeo, Vanessa Cropley, Andrew Zalesky, and others. High\sphinxhyphen{}resolution connectomic fingerprints: mapping neural identity and behavior. \sphinxstyleemphasis{NeuroImage}, 229:117695, 2021.
%
\end{footnote} (\sphinxhref{https://github.com/sina-mansour/neural-identity}{more information})

\item {} 
\sphinxAtStartPar
\sphinxhref{https://github.com/yetianmed/subcortex}{Melbourne subcortical atlas} (see Tian \sphinxstyleemphasis{et al.}%
\begin{footnote}[7]\sphinxAtStartFootnote
Ye Tian, Daniel S Margulies, Michael Breakspear, and Andrew Zalesky. Topographic organization of the human subcortex unveiled with functional connectivity gradients. \sphinxstyleemphasis{Nature neuroscience}, 23(11):1421–1432, 2020.
%
\end{footnote})

\end{itemize}




\subsection{Reproducible NeuroImaging Visualizations}
\label{\detokenize{chapters/03/03c_visualization-examples:reproducible-neuroimaging-visualizations}}
\sphinxAtStartPar
With the directory structure in place and example neuroimaging data made available, we can now turn our attention to creating a variety of code\sphinxhyphen{}based visualizations.

\sphinxAtStartPar
These examples are designed to illustrate best practices for reproducible neuroimaging visualization workflows and demonstrate how to effectively explore and present brain imaging data using code.


\subsubsection{Volume Slice Rendering}
\label{\detokenize{chapters/03/03c_visualization-examples:volume-slice-rendering}}
\sphinxAtStartPar
In this section, we’ll demonstrate how to render volume slices using data from the Human Connectome Project’s \sphinxhref{https://www.humanconnectome.org/study/hcp-young-adult/article/s1200-group-average-data-release}{\sphinxstylestrong{S1200 Group Average Data Release}}. Specifically, we’ll visualize a set of axial slices from the group\sphinxhyphen{}average T1\sphinxhyphen{}weighted image, followed by an overlay of the T2\sphinxhyphen{}weighted image using an arbitrary intensity threshold.

\sphinxAtStartPar
The images will be loaded using \sphinxstylestrong{Nibabel}, and slices will be visualized using \sphinxstylestrong{Matplotlib}. To promote reusability and reproducibility, we’ll structure the code as modular, well\sphinxhyphen{}documented functions.

\begin{sphinxuseclass}{cell}
\begin{sphinxuseclass}{tag_hide-input}
\end{sphinxuseclass}
\end{sphinxuseclass}
\begin{sphinxuseclass}{cell}\begin{sphinxVerbatimInput}

\begin{sphinxuseclass}{cell_input}
\begin{sphinxVerbatim}[commandchars=\\\{\}]
\PYG{n}{visualize\PYGZus{}multiple\PYGZus{}axial\PYGZus{}slices}\PYG{p}{(}\PYG{l+s+sa}{f}\PYG{l+s+s2}{\PYGZdq{}}\PYG{l+s+si}{\PYGZob{}}\PYG{n}{working\PYGZus{}directory}\PYG{l+s+si}{\PYGZcb{}}\PYG{l+s+s2}{/data/S1200\PYGZus{}AverageT1w\PYGZus{}restore.nii.gz}\PYG{l+s+s2}{\PYGZdq{}}\PYG{p}{)}
\end{sphinxVerbatim}

\end{sphinxuseclass}\end{sphinxVerbatimInput}
\begin{sphinxVerbatimOutput}

\begin{sphinxuseclass}{cell_output}
\noindent\sphinxincludegraphics{{7822aa0d7e347c977955f5a084ee0126e66fe5a47994cd2c37ca9940eb2d63a4}.png}

\end{sphinxuseclass}\end{sphinxVerbatimOutput}

\end{sphinxuseclass}
\begin{sphinxuseclass}{cell}
\begin{sphinxuseclass}{tag_hide-input}
\end{sphinxuseclass}
\end{sphinxuseclass}
\begin{sphinxuseclass}{cell}\begin{sphinxVerbatimInput}

\begin{sphinxuseclass}{cell_input}
\begin{sphinxVerbatim}[commandchars=\\\{\}]
\PYG{c+c1}{\PYGZsh{} Visualize T1 weighted image, and overlay T2w on top of it}
\PYG{n}{visualize\PYGZus{}multiple\PYGZus{}axial\PYGZus{}slices\PYGZus{}with\PYGZus{}overlay}\PYG{p}{(}
    \PYG{l+s+sa}{f}\PYG{l+s+s2}{\PYGZdq{}}\PYG{l+s+si}{\PYGZob{}}\PYG{n}{working\PYGZus{}directory}\PYG{l+s+si}{\PYGZcb{}}\PYG{l+s+s2}{/data/S1200\PYGZus{}AverageT1w\PYGZus{}restore.nii.gz}\PYG{l+s+s2}{\PYGZdq{}}\PYG{p}{,}
    \PYG{l+s+sa}{f}\PYG{l+s+s2}{\PYGZdq{}}\PYG{l+s+si}{\PYGZob{}}\PYG{n}{working\PYGZus{}directory}\PYG{l+s+si}{\PYGZcb{}}\PYG{l+s+s2}{/data/S1200\PYGZus{}AverageT2w\PYGZus{}restore.nii.gz}\PYG{l+s+s2}{\PYGZdq{}}
\PYG{p}{)}
\end{sphinxVerbatim}

\end{sphinxuseclass}\end{sphinxVerbatimInput}
\begin{sphinxVerbatimOutput}

\begin{sphinxuseclass}{cell_output}
\noindent\sphinxincludegraphics{{dde6f6a44441d08ac47d8f5db69c5459a478f0dc09feea678495a3d983fe8b9a}.png}

\end{sphinxuseclass}\end{sphinxVerbatimOutput}

\end{sphinxuseclass}

\subsubsection{Surface\sphinxhyphen{}based Visualizations}
\label{\detokenize{chapters/03/03c_visualization-examples:surface-based-visualizations}}
\sphinxAtStartPar
In the examples below, we will work with surface\sphinxhyphen{}based data from the Human Connectome Project’s \sphinxhref{https://www.humanconnectome.org/study/hcp-young-adult/article/s1200-group-average-data-release}{\sphinxstylestrong{S1200 Group Average Data Release}}, specifically using the fsLR template surface coordinates.

\sphinxAtStartPar
We’ll visualize the average cortical curvature mapped onto the cortical surface to highlight anatomical landmarks and folding patterns.

\begin{sphinxuseclass}{cell}
\begin{sphinxuseclass}{tag_hide-input}
\end{sphinxuseclass}
\end{sphinxuseclass}
\begin{sphinxuseclass}{cell}
\begin{sphinxuseclass}{tag_hide-input}
\end{sphinxuseclass}
\end{sphinxuseclass}
\begin{sphinxuseclass}{cell}\begin{sphinxVerbatimInput}

\begin{sphinxuseclass}{cell_input}
\begin{sphinxVerbatim}[commandchars=\\\{\}]
\PYG{c+c1}{\PYGZsh{} Plot curvature dscalar file in gray scale}
\PYG{n}{dscalar\PYGZus{}file} \PYG{o}{=} \PYG{l+s+sa}{f}\PYG{l+s+s2}{\PYGZdq{}}\PYG{l+s+si}{\PYGZob{}}\PYG{n}{working\PYGZus{}directory}\PYG{l+s+si}{\PYGZcb{}}\PYG{l+s+s2}{/data/S1200.curvature\PYGZus{}MSMAll.32k\PYGZus{}fs\PYGZus{}LR.dscalar.nii}\PYG{l+s+s2}{\PYGZdq{}}

\PYG{c+c1}{\PYGZsh{} Create a figure and axis for the plot}
\PYG{n}{fig}\PYG{p}{,} \PYG{n}{ax} \PYG{o}{=} \PYG{n}{plt}\PYG{o}{.}\PYG{n}{subplots}\PYG{p}{(}\PYG{n}{figsize}\PYG{o}{=}\PYG{p}{(}\PYG{l+m+mi}{12}\PYG{p}{,} \PYG{l+m+mi}{9}\PYG{p}{)}\PYG{p}{)}

\PYG{c+c1}{\PYGZsh{} Plot the dscalar file with Cerebro}
\PYG{n}{plot\PYGZus{}dscalar\PYGZus{}with\PYGZus{}cerebro}\PYG{p}{(}\PYG{n}{dscalar\PYGZus{}file}\PYG{p}{,} \PYG{n}{fig}\PYG{o}{=}\PYG{n}{fig}\PYG{p}{,} \PYG{n}{ax}\PYG{o}{=}\PYG{n}{ax}\PYG{p}{,} \PYG{n}{colormap}\PYG{o}{=}\PYG{n}{cc}\PYG{o}{.}\PYG{n}{cm}\PYG{o}{.}\PYG{n}{fire}\PYG{p}{,} \PYG{n}{show\PYGZus{}colorbar}\PYG{o}{=}\PYG{k+kc}{True}\PYG{p}{,} \PYG{n}{colorbar\PYGZus{}format}\PYG{o}{=}\PYG{l+s+s1}{\PYGZsq{}}\PYG{l+s+si}{\PYGZpc{}.2f}\PYG{l+s+s1}{\PYGZsq{}}\PYG{p}{)}
\end{sphinxVerbatim}

\end{sphinxuseclass}\end{sphinxVerbatimInput}
\begin{sphinxVerbatimOutput}

\begin{sphinxuseclass}{cell_output}
\noindent\sphinxincludegraphics{{800cb6f9b95d79ee9e00677e9c91929a0a232585709634b0e1e8f3cc9d652c19}.png}

\end{sphinxuseclass}\end{sphinxVerbatimOutput}

\end{sphinxuseclass}

\subsubsection{Volume\sphinxhyphen{}to\sphinxhyphen{}Surface Transformation}
\label{\detokenize{chapters/03/03c_visualization-examples:volume-to-surface-transformation}}
\sphinxAtStartPar
In this example, we demonstrate how to perform a volume\sphinxhyphen{}to\sphinxhyphen{}surface transformation using the \sphinxhref{https://github.com/yetianmed/subcortex}{Melbourne Subcortical Atlas}, specifically from its Scale 1 parcellation (available from \sphinxhref{https://github.com/yetianmed/subcortex/tree/master/Group-Parcellation/3T/Subcortex-Only}{\sphinxcode{\sphinxupquote{Tian\_Subcortex\_S1\_3T.nii.gz}}}).

\sphinxAtStartPar
We will generate and render a separate cortical surface visualization for each labeled region in the atlas.

\sphinxAtStartPar
This transformation and rendering will be performed using tools from the \sphinxstylestrong{Cerebro Brain Viewer}, which provides convenient functionality for mapping volumetric data onto surface meshes.

\begin{sphinxuseclass}{cell}
\begin{sphinxuseclass}{tag_hide-input}
\end{sphinxuseclass}
\end{sphinxuseclass}
\begin{sphinxuseclass}{cell}\begin{sphinxVerbatimInput}

\begin{sphinxuseclass}{cell_input}
\begin{sphinxVerbatim}[commandchars=\\\{\}]
\PYG{c+c1}{\PYGZsh{} Create a figure and axis for the plot}
\PYG{n}{fig}\PYG{p}{,} \PYG{n}{ax} \PYG{o}{=} \PYG{n}{plt}\PYG{o}{.}\PYG{n}{subplots}\PYG{p}{(}\PYG{n}{figsize}\PYG{o}{=}\PYG{p}{(}\PYG{l+m+mi}{10}\PYG{p}{,} \PYG{l+m+mi}{9}\PYG{p}{)}\PYG{p}{)}

\PYG{c+c1}{\PYGZsh{} Suppress low\PYGZhy{}level output}
\PYG{n}{old\PYGZus{}stdout}\PYG{p}{,} \PYG{n}{old\PYGZus{}stderr} \PYG{o}{=} \PYG{n}{suppress\PYGZus{}c\PYGZus{}output}\PYG{p}{(}\PYG{p}{)}

\PYG{k}{try}\PYG{p}{:}
    \PYG{c+c1}{\PYGZsh{} Plot the glass brain with the subcortical atlas}
    \PYG{n}{plot\PYGZus{}glass\PYGZus{}brain\PYGZus{}and\PYGZus{}atlas\PYGZus{}with\PYGZus{}cerebro}\PYG{p}{(}
        \PYG{n}{ax}\PYG{o}{=}\PYG{n}{ax}\PYG{p}{,} \PYG{n}{atlas\PYGZus{}file}\PYG{o}{=}\PYG{l+s+s2}{\PYGZdq{}}\PYG{l+s+s2}{data/Tian\PYGZus{}Subcortex\PYGZus{}S1\PYGZus{}3T.nii.gz}\PYG{l+s+s2}{\PYGZdq{}}\PYG{p}{,} \PYG{n}{view}\PYG{o}{=}\PYG{p}{(}\PYG{p}{(}\PYG{l+m+mi}{250}\PYG{p}{,} \PYG{l+m+mi}{350}\PYG{p}{,} \PYG{l+m+mi}{0}\PYG{p}{)}\PYG{p}{,} \PYG{k+kc}{None}\PYG{p}{,} \PYG{k+kc}{None}\PYG{p}{,} \PYG{k+kc}{None}\PYG{p}{)}\PYG{p}{,} \PYG{n}{surface}\PYG{o}{=}\PYG{l+s+s1}{\PYGZsq{}}\PYG{l+s+s1}{inflated}\PYG{l+s+s1}{\PYGZsq{}}\PYG{p}{,} \PYG{n}{glass\PYGZus{}color}\PYG{o}{=}\PYG{p}{(}\PYG{l+m+mf}{0.9}\PYG{p}{,} \PYG{l+m+mf}{0.9}\PYG{p}{,} \PYG{l+m+mf}{0.9}\PYG{p}{,} \PYG{l+m+mf}{0.2}\PYG{p}{)}
    \PYG{p}{)}
\PYG{k}{finally}\PYG{p}{:}
    \PYG{c+c1}{\PYGZsh{} Restore the original stdout and stderr}
    \PYG{n}{restore\PYGZus{}c\PYGZus{}output}\PYG{p}{(}\PYG{n}{old\PYGZus{}stdout}\PYG{p}{,} \PYG{n}{old\PYGZus{}stderr}\PYG{p}{)}
\end{sphinxVerbatim}

\end{sphinxuseclass}\end{sphinxVerbatimInput}
\begin{sphinxVerbatimOutput}

\begin{sphinxuseclass}{cell_output}
\noindent\sphinxincludegraphics{{f13b4061454bf16e3ec8aa8190e9783861486f21b15698d52311f0e7b2d7d9b0}.png}

\end{sphinxuseclass}\end{sphinxVerbatimOutput}

\end{sphinxuseclass}

\subsubsection{Tractography Visualization}
\label{\detokenize{chapters/03/03c_visualization-examples:tractography-visualization}}
\sphinxAtStartPar
In this section, we will visualize tractography data from the \sphinxhref{https://github.com/SlicerDMRI/ORG-Atlases?tab=readme-ov-file}{ORG fiber clustering atlas}. We will render a single fiber bundle using the Cerebro Brain Viewer, showcasing how to load and display streamline data effectively.

\begin{sphinxuseclass}{cell}
\begin{sphinxuseclass}{tag_hide-input}
\end{sphinxuseclass}
\end{sphinxuseclass}
\begin{sphinxuseclass}{cell}\begin{sphinxVerbatimInput}

\begin{sphinxuseclass}{cell_input}
\begin{sphinxVerbatim}[commandchars=\\\{\}]
\PYG{c+c1}{\PYGZsh{} Create a figure and axis for the plot}
\PYG{n}{fig}\PYG{p}{,} \PYG{n}{ax} \PYG{o}{=} \PYG{n}{plt}\PYG{o}{.}\PYG{n}{subplots}\PYG{p}{(}\PYG{n}{figsize}\PYG{o}{=}\PYG{p}{(}\PYG{l+m+mi}{10}\PYG{p}{,} \PYG{l+m+mi}{9}\PYG{p}{)}\PYG{p}{)}

\PYG{c+c1}{\PYGZsh{} Suppress low\PYGZhy{}level output}
\PYG{n}{old\PYGZus{}stdout}\PYG{p}{,} \PYG{n}{old\PYGZus{}stderr} \PYG{o}{=} \PYG{n}{suppress\PYGZus{}c\PYGZus{}output}\PYG{p}{(}\PYG{p}{)}

\PYG{k}{try}\PYG{p}{:}
    \PYG{c+c1}{\PYGZsh{} Plot the glass brain with the subcortical atlas}
    \PYG{n}{plot\PYGZus{}glass\PYGZus{}brain\PYGZus{}and\PYGZus{}tractography\PYGZus{}with\PYGZus{}cerebro}\PYG{p}{(}
        \PYG{n}{ax}\PYG{o}{=}\PYG{n}{ax}\PYG{p}{,} \PYG{c+c1}{\PYGZsh{} axis to render the brain view on}
        \PYG{n}{tract\PYGZus{}file}\PYG{o}{=}\PYG{l+s+s2}{\PYGZdq{}}\PYG{l+s+s2}{data/T\PYGZus{}SLF\PYGZhy{}III\PYGZhy{}cluster\PYGZus{}00209.tck}\PYG{l+s+s2}{\PYGZdq{}}\PYG{p}{,} \PYG{c+c1}{\PYGZsh{} path to the tractography file}
        \PYG{n}{view}\PYG{o}{=}\PYG{p}{(}\PYG{p}{(}\PYG{l+m+mi}{400}\PYG{p}{,} \PYG{l+m+mi}{150}\PYG{p}{,} \PYG{l+m+mi}{100}\PYG{p}{)}\PYG{p}{,} \PYG{k+kc}{None}\PYG{p}{,} \PYG{k+kc}{None}\PYG{p}{,} \PYG{k+kc}{None}\PYG{p}{)}\PYG{p}{,} \PYG{c+c1}{\PYGZsh{} camera view configuration for the brain viewer (a view from the right side, slightly tilted towards the frontal superior part of the brain)}
        \PYG{n}{surface}\PYG{o}{=}\PYG{l+s+s1}{\PYGZsq{}}\PYG{l+s+s1}{inflated}\PYG{l+s+s1}{\PYGZsq{}}\PYG{p}{,} \PYG{n}{glass\PYGZus{}color}\PYG{o}{=}\PYG{p}{(}\PYG{l+m+mf}{0.9}\PYG{p}{,} \PYG{l+m+mf}{0.9}\PYG{p}{,} \PYG{l+m+mf}{0.9}\PYG{p}{,} \PYG{l+m+mf}{0.2}\PYG{p}{)}
    \PYG{p}{)}
\PYG{k}{finally}\PYG{p}{:}
    \PYG{c+c1}{\PYGZsh{} Restore the original stdout and stderr}
    \PYG{n}{restore\PYGZus{}c\PYGZus{}output}\PYG{p}{(}\PYG{n}{old\PYGZus{}stdout}\PYG{p}{,} \PYG{n}{old\PYGZus{}stderr}\PYG{p}{)}

\PYG{c+c1}{\PYGZsh{} Add text to the plot}
\PYG{n}{ax}\PYG{o}{.}\PYG{n}{text}\PYG{p}{(}\PYG{l+m+mf}{0.5}\PYG{p}{,} \PYG{l+m+mf}{0.95}\PYG{p}{,} \PYG{l+s+s2}{\PYGZdq{}}\PYG{l+s+s2}{Tractography Streamlines (SLF\PYGZhy{}III)}\PYG{l+s+s2}{\PYGZdq{}}\PYG{p}{,} \PYG{n}{transform}\PYG{o}{=}\PYG{n}{ax}\PYG{o}{.}\PYG{n}{transAxes}\PYG{p}{,} \PYG{n}{fontsize}\PYG{o}{=}\PYG{l+m+mi}{16}\PYG{p}{,} \PYG{n}{ha}\PYG{o}{=}\PYG{l+s+s1}{\PYGZsq{}}\PYG{l+s+s1}{center}\PYG{l+s+s1}{\PYGZsq{}}\PYG{p}{,} \PYG{n}{va}\PYG{o}{=}\PYG{l+s+s1}{\PYGZsq{}}\PYG{l+s+s1}{center}\PYG{l+s+s1}{\PYGZsq{}}\PYG{p}{,} \PYG{n}{color}\PYG{o}{=}\PYG{l+s+s1}{\PYGZsq{}}\PYG{l+s+s1}{black}\PYG{l+s+s1}{\PYGZsq{}}\PYG{p}{)}
\end{sphinxVerbatim}

\end{sphinxuseclass}\end{sphinxVerbatimInput}
\begin{sphinxVerbatimOutput}

\begin{sphinxuseclass}{cell_output}
\begin{sphinxVerbatim}[commandchars=\\\{\}]
Text(0.5, 0.95, \PYGZsq{}Tractography Streamlines (SLF\PYGZhy{}III)\PYGZsq{})
\end{sphinxVerbatim}

\noindent\sphinxincludegraphics{{70c19abd07ae381b1c57ff419d8e49f4c9464fc41a3792da13764761cdda193e}.png}

\end{sphinxuseclass}\end{sphinxVerbatimOutput}

\end{sphinxuseclass}

\subsubsection{Brain Network Visualizations}
\label{\detokenize{chapters/03/03c_visualization-examples:brain-network-visualizations}}
\sphinxAtStartPar
This example uses functional human connectome data from Mansour \sphinxstyleemphasis{et al.}\sphinxfootnotemark[6]. We will visualize both a connectivity heatmap and a 3D brain network derived from an individual’s functional connectome, mapped onto the HCP MMP1 atlas (Glasser \sphinxstyleemphasis{et al.}\sphinxfootnotemark[4]).

\sphinxAtStartPar
The connectivity heatmap will be created using \sphinxstylestrong{Matplotlib}, while the 3D brain network visualization will be generated with the \sphinxstylestrong{Cerebro Brain Viewer}.

\begin{sphinxuseclass}{cell}
\begin{sphinxuseclass}{tag_hide-input}
\end{sphinxuseclass}
\end{sphinxuseclass}
\begin{sphinxuseclass}{cell}\begin{sphinxVerbatimInput}

\begin{sphinxuseclass}{cell_input}
\begin{sphinxVerbatim}[commandchars=\\\{\}]
\PYG{c+c1}{\PYGZsh{} Get a figure and axis for the heatmap}
\PYG{n}{fig}\PYG{p}{,} \PYG{n}{ax} \PYG{o}{=} \PYG{n}{plt}\PYG{o}{.}\PYG{n}{subplots}\PYG{p}{(}\PYG{n}{figsize}\PYG{o}{=}\PYG{p}{(}\PYG{l+m+mi}{10}\PYG{p}{,} \PYG{l+m+mi}{10}\PYG{p}{)}\PYG{p}{)}

\PYG{c+c1}{\PYGZsh{} Plot the heatmap of the functional connectivity matrix}
\PYG{n}{im} \PYG{o}{=} \PYG{n}{ax}\PYG{o}{.}\PYG{n}{imshow}\PYG{p}{(}\PYG{n}{fc\PYGZus{}matrix\PYGZus{}reordered}\PYG{p}{,} \PYG{n}{cmap}\PYG{o}{=}\PYG{n}{cc}\PYG{o}{.}\PYG{n}{cm}\PYG{o}{.}\PYG{n}{coolwarm}\PYG{p}{,} \PYG{n}{aspect}\PYG{o}{=}\PYG{l+s+s1}{\PYGZsq{}}\PYG{l+s+s1}{auto}\PYG{l+s+s1}{\PYGZsq{}}\PYG{p}{,} \PYG{n}{interpolation}\PYG{o}{=}\PYG{l+s+s1}{\PYGZsq{}}\PYG{l+s+s1}{nearest}\PYG{l+s+s1}{\PYGZsq{}}\PYG{p}{,} \PYG{n}{vmin}\PYG{o}{=}\PYG{o}{\PYGZhy{}}\PYG{l+m+mi}{1}\PYG{p}{,} \PYG{n}{vmax}\PYG{o}{=}\PYG{l+m+mi}{1}\PYG{p}{)}

\PYG{c+c1}{\PYGZsh{} Set the ticks and labels for the heatmap}
\PYG{n}{glasser\PYGZus{}label\PYGZus{}names} \PYG{o}{=} \PYG{p}{[}\PYG{n}{glasser\PYGZus{}atlas\PYGZus{}labels\PYGZus{}dict}\PYG{p}{[}\PYG{n}{x}\PYG{p}{]}\PYG{p}{[}\PYG{l+m+mi}{0}\PYG{p}{]} \PYG{k}{for} \PYG{n}{x} \PYG{o+ow}{in} \PYG{n+nb}{range}\PYG{p}{(}\PYG{l+m+mi}{1}\PYG{p}{,} \PYG{n+nb}{max}\PYG{p}{(}\PYG{n}{glasser\PYGZus{}atlas\PYGZus{}labels\PYGZus{}dict}\PYG{o}{.}\PYG{n}{keys}\PYG{p}{(}\PYG{p}{)}\PYG{p}{)} \PYG{o}{+} \PYG{l+m+mi}{1}\PYG{p}{)}\PYG{p}{]}

\PYG{c+c1}{\PYGZsh{} Add a dashed line at the middle of the heatmap to separate left and right hemispheres}
\PYG{n}{ax}\PYG{o}{.}\PYG{n}{axvline}\PYG{p}{(}\PYG{n}{x}\PYG{o}{=}\PYG{n+nb}{len}\PYG{p}{(}\PYG{n}{glasser\PYGZus{}label\PYGZus{}names}\PYG{p}{)}\PYG{o}{/}\PYG{o}{/}\PYG{l+m+mi}{2}\PYG{p}{,} \PYG{n}{color}\PYG{o}{=}\PYG{l+s+s1}{\PYGZsq{}}\PYG{l+s+s1}{gray}\PYG{l+s+s1}{\PYGZsq{}}\PYG{p}{,} \PYG{n}{linestyle}\PYG{o}{=}\PYG{l+s+s1}{\PYGZsq{}}\PYG{l+s+s1}{\PYGZhy{}\PYGZhy{}}\PYG{l+s+s1}{\PYGZsq{}}\PYG{p}{,} \PYG{n}{linewidth}\PYG{o}{=}\PYG{l+m+mi}{1}\PYG{p}{)}
\PYG{n}{ax}\PYG{o}{.}\PYG{n}{axhline}\PYG{p}{(}\PYG{n}{y}\PYG{o}{=}\PYG{n+nb}{len}\PYG{p}{(}\PYG{n}{glasser\PYGZus{}label\PYGZus{}names}\PYG{p}{)}\PYG{o}{/}\PYG{o}{/}\PYG{l+m+mi}{2}\PYG{p}{,} \PYG{n}{color}\PYG{o}{=}\PYG{l+s+s1}{\PYGZsq{}}\PYG{l+s+s1}{gray}\PYG{l+s+s1}{\PYGZsq{}}\PYG{p}{,} \PYG{n}{linestyle}\PYG{o}{=}\PYG{l+s+s1}{\PYGZsq{}}\PYG{l+s+s1}{\PYGZhy{}\PYGZhy{}}\PYG{l+s+s1}{\PYGZsq{}}\PYG{p}{,} \PYG{n}{linewidth}\PYG{o}{=}\PYG{l+m+mi}{1}\PYG{p}{)}

\PYG{c+c1}{\PYGZsh{} Add a colorbar to the heatmap}
\PYG{n}{cbar} \PYG{o}{=} \PYG{n}{fig}\PYG{o}{.}\PYG{n}{colorbar}\PYG{p}{(}\PYG{n}{im}\PYG{p}{,} \PYG{n}{ax}\PYG{o}{=}\PYG{n}{ax}\PYG{p}{,} \PYG{n}{orientation}\PYG{o}{=}\PYG{l+s+s1}{\PYGZsq{}}\PYG{l+s+s1}{vertical}\PYG{l+s+s1}{\PYGZsq{}}\PYG{p}{,} \PYG{n}{fraction}\PYG{o}{=}\PYG{l+m+mf}{0.05}\PYG{p}{,} \PYG{n}{pad}\PYG{o}{=}\PYG{l+m+mf}{0.04}\PYG{p}{)}
\PYG{n}{cbar}\PYG{o}{.}\PYG{n}{set\PYGZus{}label}\PYG{p}{(}\PYG{l+s+s1}{\PYGZsq{}}\PYG{l+s+s1}{Functional Connectivity (Pearson Correlation)}\PYG{l+s+s1}{\PYGZsq{}}\PYG{p}{,} \PYG{n}{fontsize}\PYG{o}{=}\PYG{l+m+mi}{14}\PYG{p}{)}

\PYG{c+c1}{\PYGZsh{} Set the title for the heatmap}
\PYG{n}{ax}\PYG{o}{.}\PYG{n}{set\PYGZus{}title}\PYG{p}{(}\PYG{l+s+s2}{\PYGZdq{}}\PYG{l+s+s2}{Functional Connectivity Matrix (Glasser Atlas)}\PYG{l+s+s2}{\PYGZdq{}}\PYG{p}{,} \PYG{n}{fontsize}\PYG{o}{=}\PYG{l+m+mi}{16}\PYG{p}{)}
\end{sphinxVerbatim}

\end{sphinxuseclass}\end{sphinxVerbatimInput}
\begin{sphinxVerbatimOutput}

\begin{sphinxuseclass}{cell_output}
\begin{sphinxVerbatim}[commandchars=\\\{\}]
Text(0.5, 1.0, \PYGZsq{}Functional Connectivity Matrix (Glasser Atlas)\PYGZsq{})
\end{sphinxVerbatim}

\noindent\sphinxincludegraphics{{c92af824ad4e3703f18739d736e3b5d958cf68b4b0fb9e336d9ca25f0feb1e41}.png}

\end{sphinxuseclass}\end{sphinxVerbatimOutput}

\end{sphinxuseclass}
\begin{sphinxuseclass}{cell}
\begin{sphinxuseclass}{tag_hide-input}
\end{sphinxuseclass}
\end{sphinxuseclass}
\begin{sphinxuseclass}{cell}\begin{sphinxVerbatimInput}

\begin{sphinxuseclass}{cell_input}
\begin{sphinxVerbatim}[commandchars=\\\{\}]
\PYG{c+c1}{\PYGZsh{} Now that we have the functional connectivity matrix,}
\PYG{c+c1}{\PYGZsh{} we need the coordinates of the nodes in the glasser atlas}
\PYG{n}{node\PYGZus{}coords} \PYG{o}{=} \PYG{n}{np}\PYG{o}{.}\PYG{n}{array}\PYG{p}{(}\PYG{p}{[}\PYG{n}{lrxyz}\PYG{p}{[}\PYG{n}{surface\PYGZus{}mask}\PYG{p}{]}\PYG{p}{[}\PYG{n}{glasser\PYGZus{}atlas}\PYG{o}{.}\PYG{n}{get\PYGZus{}fdata}\PYG{p}{(}\PYG{p}{)}\PYG{p}{[}\PYG{l+m+mi}{0}\PYG{p}{]} \PYG{o}{==} \PYG{n}{x}\PYG{p}{]}\PYG{o}{.}\PYG{n}{mean}\PYG{p}{(}\PYG{n}{axis}\PYG{o}{=}\PYG{l+m+mi}{0}\PYG{p}{)} \PYG{k}{for} \PYG{n}{x} \PYG{o+ow}{in} \PYG{n+nb}{range}\PYG{p}{(}\PYG{l+m+mi}{1}\PYG{p}{,} \PYG{n+nb}{max}\PYG{p}{(}\PYG{n}{glasser\PYGZus{}atlas\PYGZus{}labels\PYGZus{}dict}\PYG{o}{.}\PYG{n}{keys}\PYG{p}{(}\PYG{p}{)}\PYG{p}{)}\PYG{o}{+}\PYG{l+m+mi}{1}\PYG{p}{)}\PYG{p}{]}\PYG{p}{)}

\PYG{c+c1}{\PYGZsh{} Let\PYGZsq{}s also extract node colors from the glasser atlas}
\PYG{n}{glasser\PYGZus{}label\PYGZus{}colors} \PYG{o}{=} \PYG{p}{[}\PYG{n}{glasser\PYGZus{}atlas\PYGZus{}labels\PYGZus{}dict}\PYG{p}{[}\PYG{n}{x}\PYG{p}{]}\PYG{p}{[}\PYG{l+m+mi}{1}\PYG{p}{]} \PYG{k}{for} \PYG{n}{x} \PYG{o+ow}{in} \PYG{n+nb}{range}\PYG{p}{(}\PYG{l+m+mi}{1}\PYG{p}{,} \PYG{n+nb}{max}\PYG{p}{(}\PYG{n}{glasser\PYGZus{}atlas\PYGZus{}labels\PYGZus{}dict}\PYG{o}{.}\PYG{n}{keys}\PYG{p}{(}\PYG{p}{)}\PYG{p}{)} \PYG{o}{+} \PYG{l+m+mi}{1}\PYG{p}{)}\PYG{p}{]}

\PYG{c+c1}{\PYGZsh{} Now let\PYGZsq{}s threshold the functional connectivity matrix to only keep the strongest connections}
\PYG{n}{threshold} \PYG{o}{=} \PYG{l+m+mf}{0.7}  \PYG{c+c1}{\PYGZsh{} threshold for the functional connectivity matrix}
\PYG{n}{fc\PYGZus{}matrix\PYGZus{}thresholded} \PYG{o}{=} \PYG{n}{np}\PYG{o}{.}\PYG{n}{where}\PYG{p}{(}\PYG{n}{np}\PYG{o}{.}\PYG{n}{abs}\PYG{p}{(}\PYG{n}{fc\PYGZus{}matrix\PYGZus{}reordered}\PYG{p}{)} \PYG{o}{\PYGZgt{}} \PYG{n}{threshold}\PYG{p}{,} \PYG{n}{fc\PYGZus{}matrix\PYGZus{}reordered}\PYG{p}{,} \PYG{l+m+mi}{0}\PYG{p}{)}
\PYG{c+c1}{\PYGZsh{} also remove self\PYGZhy{}connections}
\PYG{n}{np}\PYG{o}{.}\PYG{n}{fill\PYGZus{}diagonal}\PYG{p}{(}\PYG{n}{fc\PYGZus{}matrix\PYGZus{}thresholded}\PYG{p}{,} \PYG{l+m+mi}{0}\PYG{p}{)}

\PYG{c+c1}{\PYGZsh{} Node sizes can be set based on the degree of each node}
\PYG{n}{node\PYGZus{}degrees} \PYG{o}{=} \PYG{n}{np}\PYG{o}{.}\PYG{n}{sum}\PYG{p}{(}\PYG{n}{np}\PYG{o}{.}\PYG{n}{abs}\PYG{p}{(}\PYG{n}{fc\PYGZus{}matrix\PYGZus{}thresholded}\PYG{p}{)}\PYG{p}{,} \PYG{n}{axis}\PYG{o}{=}\PYG{l+m+mi}{1}\PYG{p}{)}
\PYG{n}{node\PYGZus{}radii} \PYG{o}{=} \PYG{n}{np}\PYG{o}{.}\PYG{n}{clip}\PYG{p}{(}\PYG{n}{node\PYGZus{}degrees} \PYG{o}{/} \PYG{n}{np}\PYG{o}{.}\PYG{n}{max}\PYG{p}{(}\PYG{n}{node\PYGZus{}degrees}\PYG{p}{)} \PYG{o}{*} \PYG{l+m+mi}{5}\PYG{p}{,} \PYG{l+m+mi}{1}\PYG{p}{,} \PYG{l+m+mi}{5}\PYG{p}{)}  \PYG{c+c1}{\PYGZsh{} Scale node sizes between 1 and 5}
\PYG{n}{node\PYGZus{}radii} \PYG{o}{=} \PYG{n}{np}\PYG{o}{.}\PYG{n}{repeat}\PYG{p}{(}\PYG{n}{node\PYGZus{}radii}\PYG{p}{[}\PYG{p}{:}\PYG{p}{,} \PYG{n}{np}\PYG{o}{.}\PYG{n}{newaxis}\PYG{p}{]}\PYG{p}{,} \PYG{l+m+mi}{3}\PYG{p}{,} \PYG{n}{axis}\PYG{o}{=}\PYG{l+m+mi}{1}\PYG{p}{)} \PYG{c+c1}{\PYGZsh{} Make node radii along 3 dimensions (N x 3)}

\PYG{c+c1}{\PYGZsh{} Edge radii can be set based on the strength of the connections}
\PYG{n}{edge\PYGZus{}radii} \PYG{o}{=} \PYG{n}{np}\PYG{o}{.}\PYG{n}{clip}\PYG{p}{(}\PYG{n}{np}\PYG{o}{.}\PYG{n}{abs}\PYG{p}{(}\PYG{n}{fc\PYGZus{}matrix\PYGZus{}thresholded}\PYG{p}{)} \PYG{o}{/} \PYG{n}{np}\PYG{o}{.}\PYG{n}{max}\PYG{p}{(}\PYG{n}{np}\PYG{o}{.}\PYG{n}{abs}\PYG{p}{(}\PYG{n}{fc\PYGZus{}matrix\PYGZus{}thresholded}\PYG{p}{)}\PYG{p}{)}\PYG{p}{,} \PYG{l+m+mf}{0.2}\PYG{p}{,} \PYG{l+m+mi}{1}\PYG{p}{)}  \PYG{c+c1}{\PYGZsh{} Scale edge sizes between 0.2 and 1}

\PYG{c+c1}{\PYGZsh{} Create a figure and axis for the plot}
\PYG{n}{fig}\PYG{p}{,} \PYG{n}{ax} \PYG{o}{=} \PYG{n}{plt}\PYG{o}{.}\PYG{n}{subplots}\PYG{p}{(}\PYG{n}{figsize}\PYG{o}{=}\PYG{p}{(}\PYG{l+m+mi}{10}\PYG{p}{,} \PYG{l+m+mi}{9}\PYG{p}{)}\PYG{p}{)}

\PYG{c+c1}{\PYGZsh{} Suppress low\PYGZhy{}level output}
\PYG{n}{old\PYGZus{}stdout}\PYG{p}{,} \PYG{n}{old\PYGZus{}stderr} \PYG{o}{=} \PYG{n}{suppress\PYGZus{}c\PYGZus{}output}\PYG{p}{(}\PYG{p}{)}

\PYG{k}{try}\PYG{p}{:}
    \PYG{c+c1}{\PYGZsh{} Plot the glass brain with the subcortical atlas}
    \PYG{n}{plot\PYGZus{}glass\PYGZus{}brain\PYGZus{}and\PYGZus{}network\PYGZus{}with\PYGZus{}cerebro}\PYG{p}{(}
        \PYG{n}{ax}\PYG{o}{=}\PYG{n}{ax}\PYG{p}{,} \PYG{c+c1}{\PYGZsh{} axis to render the brain view on}
        \PYG{n}{adjacency\PYGZus{}matrix}\PYG{o}{=}\PYG{n}{fc\PYGZus{}matrix\PYGZus{}thresholded}\PYG{p}{,}  \PYG{c+c1}{\PYGZsh{} The adjacency matrix of the network}
        \PYG{n}{node\PYGZus{}coords}\PYG{o}{=}\PYG{n}{node\PYGZus{}coords}\PYG{p}{,}  \PYG{c+c1}{\PYGZsh{} The coordinates of the nodes in the network}
        \PYG{n}{node\PYGZus{}colors}\PYG{o}{=}\PYG{n}{glasser\PYGZus{}label\PYGZus{}colors}\PYG{p}{,}  \PYG{c+c1}{\PYGZsh{} The colors of the nodes in the network}
        \PYG{n}{node\PYGZus{}radii}\PYG{o}{=}\PYG{n}{node\PYGZus{}radii}\PYG{p}{,}  \PYG{c+c1}{\PYGZsh{} The radii of the nodes in the network}
        \PYG{n}{edge\PYGZus{}radii}\PYG{o}{=}\PYG{n}{edge\PYGZus{}radii}\PYG{p}{,}  \PYG{c+c1}{\PYGZsh{} The radii of the edges in the network}
        \PYG{n}{view}\PYG{o}{=}\PYG{p}{(}\PYG{p}{(}\PYG{l+m+mi}{400}\PYG{p}{,} \PYG{l+m+mi}{150}\PYG{p}{,} \PYG{l+m+mi}{100}\PYG{p}{)}\PYG{p}{,} \PYG{k+kc}{None}\PYG{p}{,} \PYG{k+kc}{None}\PYG{p}{,} \PYG{k+kc}{None}\PYG{p}{)}\PYG{p}{,} \PYG{c+c1}{\PYGZsh{} camera view configuration for the brain viewer (a view from the right side, slightly tilted towards the frontal superior part of the brain)}
        \PYG{n}{surface}\PYG{o}{=}\PYG{l+s+s1}{\PYGZsq{}}\PYG{l+s+s1}{inflated}\PYG{l+s+s1}{\PYGZsq{}}\PYG{p}{,} \PYG{n}{glass\PYGZus{}color}\PYG{o}{=}\PYG{p}{(}\PYG{l+m+mf}{0.9}\PYG{p}{,} \PYG{l+m+mf}{0.9}\PYG{p}{,} \PYG{l+m+mf}{0.9}\PYG{p}{,} \PYG{l+m+mf}{0.2}\PYG{p}{)}
    \PYG{p}{)}
\PYG{k}{finally}\PYG{p}{:}
    \PYG{c+c1}{\PYGZsh{} Restore the original stdout and stderr}
    \PYG{n}{restore\PYGZus{}c\PYGZus{}output}\PYG{p}{(}\PYG{n}{old\PYGZus{}stdout}\PYG{p}{,} \PYG{n}{old\PYGZus{}stderr}\PYG{p}{)}
\end{sphinxVerbatim}

\end{sphinxuseclass}\end{sphinxVerbatimInput}
\begin{sphinxVerbatimOutput}

\begin{sphinxuseclass}{cell_output}
\begin{sphinxVerbatim}[commandchars=\\\{\}]
(206,) (206, 2) (206, 4)
\end{sphinxVerbatim}

\noindent\sphinxincludegraphics{{4544c35b0d670b50dcded3f42464be94407e45a48fc480213207468eb0b15d85}.png}

\end{sphinxuseclass}\end{sphinxVerbatimOutput}

\end{sphinxuseclass}

\bigskip\hrule\bigskip



\subsection{📑 References}
\label{\detokenize{chapters/03/03c_visualization-examples:references}}

\bigskip\hrule\bigskip


\sphinxstepscope


\chapter{📑 References}
\label{\detokenize{chapters/04/references:references}}\label{\detokenize{chapters/04/references::doc}}
\begin{sphinxthebibliography}{10}
\bibitem[1]{chapters/04/references:id2}
\sphinxAtStartPar
Marcus R Munafò, Brian A Nosek, Dorothy VM Bishop, Katherine S Button, Christopher D Chambers, Nathalie Percie du Sert, Uri Simonsohn, Eric\sphinxhyphen{}Jan Wagenmakers, Jennifer J Ware, and John PA Ioannidis. A manifesto for reproducible science. \sphinxstyleemphasis{Nature human behaviour}, 1(1):0021, 2017.
\bibitem[2]{chapters/04/references:id3}
\sphinxAtStartPar
Krzysztof J Gorgolewski and Russell A Poldrack. A practical guide for improving transparency and reproducibility in neuroimaging research. \sphinxstyleemphasis{PLoS biology}, 14(7):e1002506, 2016.
\bibitem[3]{chapters/04/references:id4}
\sphinxAtStartPar
Rotem Botvinik\sphinxhyphen{}Nezer and Tor D Wager. Reproducibility in neuroimaging analysis: challenges and solutions. \sphinxstyleemphasis{Biological Psychiatry: Cognitive Neuroscience and Neuroimaging}, 8(8):780–788, 2023.
\bibitem[4]{chapters/04/references:id5}
\sphinxAtStartPar
Guiomar Niso, Rotem Botvinik\sphinxhyphen{}Nezer, Stefan Appelhoff, Alejandro De La Vega, Oscar Esteban, Joset A Etzel, Karolina Finc, Melanie Ganz, Rémi Gau, Yaroslav O Halchenko, and others. Open and reproducible neuroimaging: from study inception to publication. \sphinxstyleemphasis{NeuroImage}, 263:119623, 2022.
\bibitem[5]{chapters/04/references:id6}
\sphinxAtStartPar
Greg Wilson, Jennifer Bryan, Karen Cranston, Justin Kitzes, Lex Nederbragt, and Tracy K Teal. Good enough practices in scientific computing. \sphinxstyleemphasis{PLoS computational biology}, 13(6):e1005510, 2017.
\bibitem[6]{chapters/04/references:id7}
\sphinxAtStartPar
Giovanni Petri, Paul Expert, Federico Turkheimer, Robin Carhart\sphinxhyphen{}Harris, David Nutt, Peter J Hellyer, and Francesco Vaccarino. Homological scaffolds of brain functional networks. \sphinxstyleemphasis{Journal of The Royal Society Interface}, 11(101):20140873, 2014.
\bibitem[7]{chapters/04/references:id8}
\sphinxAtStartPar
Taylor Bolt, Shiyu Wang, Jason S Nomi, Roni Setton, Benjamin P Gold, BT Yeo, J Jean Chen, Dante Picchioni, Jeff H Duyn, R Nathan Spreng, and others. Autonomic physiological coupling of the global fmri signal. \sphinxstyleemphasis{Nature Neuroscience}, pages 1–9, 2025.
\bibitem[8]{chapters/04/references:id9}
\sphinxAtStartPar
Julia M Huntenburg, Pierre\sphinxhyphen{}Louis Bazin, and Daniel S Margulies. Large\sphinxhyphen{}scale gradients in human cortical organization. \sphinxstyleemphasis{Trends in cognitive sciences}, 22(1):21–31, 2018.
\bibitem[9]{chapters/04/references:id10}
\sphinxAtStartPar
Daniel S Margulies, Satrajit S Ghosh, Alexandros Goulas, Marcel Falkiewicz, Julia M Huntenburg, Georg Langs, Gleb Bezgin, Simon B Eickhoff, F Xavier Castellanos, Michael Petrides, and others. Situating the default\sphinxhyphen{}mode network along a principal gradient of macroscale cortical organization. \sphinxstyleemphasis{Proceedings of the National Academy of Sciences}, 113(44):12574–12579, 2016.
\bibitem[10]{chapters/04/references:id11}
\sphinxAtStartPar
Hari McGrath, Hitten P Zaveri, Evan Collins, Tamara Jafar, Omar Chishti, Sami Obaid, Alexander Ksendzovsky, Kun Wu, Xenophon Papademetris, and Dennis D Spencer. High\sphinxhyphen{}resolution cortical parcellation based on conserved brain landmarks for localization of multimodal data to the nearest centimeter. \sphinxstyleemphasis{Scientific reports}, 12(1):18778, 2022.
\bibitem[11]{chapters/04/references:id12}
\sphinxAtStartPar
Simon R Cox, Stuart J Ritchie, Elliot M Tucker\sphinxhyphen{}Drob, David C Liewald, Saskia P Hagenaars, Gail Davies, Joanna M Wardlaw, Catharine R Gale, Mark E Bastin, and Ian J Deary. Ageing and brain white matter structure in 3,513 uk biobank participants. \sphinxstyleemphasis{Nature communications}, 7(1):13629, 2016.
\bibitem[12]{chapters/04/references:id13}
\sphinxAtStartPar
Paul Klauser, Vanessa L Cropley, Philipp S Baumann, Jinglei Lv, Pascal Steullet, Daniella Dwir, Yasser Alemán\sphinxhyphen{}Gómez, Meritxell Bach Cuadra, Michel Cuenod, Kim Q Do, and others. White matter alterations between brain network hubs underlie processing speed impairment in patients with schizophrenia. \sphinxstyleemphasis{Schizophrenia Bulletin Open}, 2(1):sgab033, 2021.
\bibitem[13]{chapters/04/references:id14}
\sphinxAtStartPar
Farnaz Zamani Esfahlani, Joshua Faskowitz, Jonah Slack, Bratislav Mišić, and Richard F Betzel. Local structure\sphinxhyphen{}function relationships in human brain networks across the lifespan. \sphinxstyleemphasis{Nature communications}, 13(1):2053, 2022.
\bibitem[14]{chapters/04/references:id15}
\sphinxAtStartPar
Caio Seguin, Olaf Sporns, and Andrew Zalesky. Brain network communication: concepts, models and applications. \sphinxstyleemphasis{Nature reviews neuroscience}, 24(9):557–574, 2023.
\bibitem[15]{chapters/04/references:id16}
\sphinxAtStartPar
Sidhant Chopra, Loïc Labache, Elvisha Dhamala, Edwina R Orchard, and Avram Holmes. A practical guide for generating reproducible and programmatic neuroimaging visualizations. \sphinxstyleemphasis{Aperture Neuro}, 3:1–20, 2023.
\bibitem[16]{chapters/04/references:id17}
\sphinxAtStartPar
Cyril R Pernet and Christopher R Madan. Data visualization for inference in tomographic brain imaging. \sphinxstyleemphasis{European Journal of Neuroscience}, 2019.
\bibitem[17]{chapters/04/references:id18}
\sphinxAtStartPar
Christopher Rorden. From mricro to mricron: the evolution of neuroimaging visualization tools. \sphinxstyleemphasis{Neuropsychologia}, pages 109067, 2025.
\bibitem[18]{chapters/04/references:id19}
\sphinxAtStartPar
Maxime Chamberland, Charles Poirier, Tom Hendriks, Dmitri Shastin, Anna Vilanova, and Alexander Leemans. Tractography visualization. In \sphinxstyleemphasis{Handbook of Diffusion MR Tractography}, pages 381–393. Elsevier, 2025.
\bibitem[19]{chapters/04/references:id20}
\sphinxAtStartPar
Sina Mansour, Ye Tian, BT Thomas Yeo, Vanessa Cropley, Andrew Zalesky, and others. High\sphinxhyphen{}resolution connectomic fingerprints: mapping neural identity and behavior. \sphinxstyleemphasis{NeuroImage}, 229:117695, 2021.
\bibitem[20]{chapters/04/references:id21}
\sphinxAtStartPar
Matthew F Glasser, Timothy S Coalson, Emma C Robinson, Carl D Hacker, John Harwell, Essa Yacoub, Kamil Ugurbil, Jesper Andersson, Christian F Beckmann, Mark Jenkinson, and others. A multi\sphinxhyphen{}modal parcellation of human cerebral cortex. \sphinxstyleemphasis{Nature}, 536(7615):171–178, 2016.
\bibitem[21]{chapters/04/references:id22}
\sphinxAtStartPar
Matthew F Glasser, Stamatios N Sotiropoulos, J Anthony Wilson, Timothy S Coalson, Bruce Fischl, Jesper L Andersson, Junqian Xu, Saad Jbabdi, Matthew Webster, Jonathan R Polimeni, and others. The minimal preprocessing pipelines for the human connectome project. \sphinxstyleemphasis{Neuroimage}, 80:105–124, 2013.
\bibitem[22]{chapters/04/references:id23}
\sphinxAtStartPar
David C Van Essen, Kamil Ugurbil, Edward Auerbach, Deanna Barch, Timothy EJ Behrens, Richard Bucholz, Acer Chang, Liyong Chen, Maurizio Corbetta, Sandra W Curtiss, and others. The human connectome project: a data acquisition perspective. \sphinxstyleemphasis{Neuroimage}, 62(4):2222–2231, 2012.
\bibitem[23]{chapters/04/references:id24}
\sphinxAtStartPar
Daniel S Marcus, Michael P Harms, Abraham Z Snyder, Mark Jenkinson, J Anthony Wilson, Matthew F Glasser, Deanna M Barch, Kevin A Archie, Gregory C Burgess, Mohana Ramaratnam, and others. Human connectome project informatics: quality control, database services, and data visualization. \sphinxstyleemphasis{Neuroimage}, 80:202–219, 2013.
\bibitem[24]{chapters/04/references:id25}
\sphinxAtStartPar
Fan Zhang, Ye Wu, Isaiah Norton, Laura Rigolo, Yogesh Rathi, Nikos Makris, and Lauren J O'Donnell. An anatomically curated fiber clustering white matter atlas for consistent white matter tract parcellation across the lifespan. \sphinxstyleemphasis{Neuroimage}, 179:429–447, 2018.
\bibitem[25]{chapters/04/references:id26}
\sphinxAtStartPar
Ye Tian, Daniel S Margulies, Michael Breakspear, and Andrew Zalesky. Topographic organization of the human subcortex unveiled with functional connectivity gradients. \sphinxstyleemphasis{Nature neuroscience}, 23(11):1421–1432, 2020.
\end{sphinxthebibliography}







\renewcommand{\indexname}{Index}
\printindex
\end{document}